\input{./src/main.sty}
% Additional SI unit for Fahrenheit
\DeclareSIUnit\fahrenheit{\degree F}

\begin{document}

% Include title page
\input{./src/titlepage.tex}

\pagebreak

\section*{4.1.2}

Suppose that there are two quantum-mechanically measurable
quantities, $c$ with associated operator
$\hat C$ , and $d$ with associated operator $\hat D$ . In particular,
operator $\hat C$ has two eigenvectors $|\phi_1\rangle$ and $|\phi_2\rangle$,
and similarly operator $\hat D$ has two eigenvectors $|\psi_1\rangle$
and $|\psi_2\rangle$. The relation between the
eigenvectors is

\begin{align*}
  |\phi_1\rangle &= \frac{1}{5}(3|\psi_1\rangle + 4|\psi_2\rangle) \\
  |\phi_2\rangle &= \frac{1}{5}(4|\psi_1\rangle - 3|\psi_2\rangle)
\end{align*}

Suppose a measurement is made of the quantity $c$, and the system is
measured to be in state $|\phi_1\rangle$.
Then a measurement is made of quantity $d$, and following that the
quantity $c$ is again measured. What
is the probability (expressed as a fraction) that the system will be
found in state $|\psi_1\rangle$ on this second
measurement of $c$?

[Note: this is really a problem in quantum
  mechanical measurement discussed in
the previous Chapter, but is a good exercise in the use of the Dirac notation.]

\boxedanswer{

  \begin{align*}
    \hat C |\phi_1\rangle &= c|\phi_1\rangle \\
    \hat C |\phi_2\rangle &= c|\phi_2\rangle \\
    \hat D |\psi_1\rangle &= d|\phi_1\rangle \\
    \hat D |\psi_2\rangle &= d|\phi_2\rangle \\
    | \phi_1 \rangle &= \frac{1}{5}(3|\psi_1\rangle + 4 | \psi_2 \rangle) \\
    |\psi_1\rangle &= \frac{5}{3}| \phi_1 \rangle - \frac{4}{3}|
    \psi_2 \rangle \\
    |\phi_2\rangle &= \frac{1}{5}(4|\psi_1\rangle - 3|\psi_2\rangle) \\
    |\psi_2\rangle &= \frac{4}{3}|\psi_1\rangle - \frac{5}{3}|\phi_2\rangle \\
    |\psi_1\rangle &= \frac{5}{3}| \phi_1 \rangle - \frac{4}{3}
    \left(\frac{4}{3}|\psi_1\rangle -
    \frac{5}{3}|\phi_2\rangle\right) \\
    |\psi_1\rangle &= \frac{5}{3}| \phi_1 \rangle
    - \frac{16}{9}|\psi_1\rangle
    + \frac{20}{9}|\phi_2\rangle \\
    \frac{25}{9}|\psi_1\rangle
    &= \frac{5}{3}| \phi_1 \rangle
    + \frac{20}{9}|\phi_2\rangle \\
    |\psi_1\rangle
    &= \frac{3}{5}| \phi_1 \rangle
    + \frac{4}{5}|\phi_2\rangle \\
    |\psi_2\rangle &= \frac{4}{3}\left( \frac{3}{5}| \phi_1 \rangle
    + \frac{4}{5}|\phi_2\rangle\right) - \frac{5}{3}|\phi_2\rangle \\
    |\psi_2\rangle &= \frac{4}{5}| \phi_1 \rangle
    - \frac{3}{5}|\phi_2\rangle \\
  \end{align*}

  The probability of getting state $\psi_1$ after measuring $d$,
  given that the system is in state $\phi_1$.:

  \begin{align*}
    P(\psi_1) &= |\langle\psi_1|\phi_1\rangle|^2 =
    \left(\frac{3}{5}\overbrace{\langle\psi_1|\psi_1\rangle}^1 + \frac{4}{5}
    \cancel{\langle\psi_1| \psi_2 \rangle}\right)^2 \\
    P(\psi_1) &= \frac{9}{25}
  \end{align*}

}

\boxedanswer{
  The probability of getting state $\psi_2$ after measuring $d$,
  given that the system is in state $\phi_1$.:

  \begin{align*}
    P(\psi_2) &= |\langle\psi_2|\phi_1\rangle|^2 =
    \left(\frac{3}{5}\cancel{\langle\psi_2|\psi_1\rangle} + \frac{4}{5}
    \overbrace{\langle\psi_2| \psi_2 \rangle}^1\right)^2 \\
    P(\psi_2) &= \frac{16}{25}
  \end{align*}

  The probability of getting state $\phi_1$ given state $\psi_1$:

  \begin{align*}
    P(\phi_1|\psi_1) &= |\langle \phi_1|\psi_1 \rangle|^2 =
    \left(\frac{3}{5}\langle \phi_1| \phi_1 \rangle
    + \frac{4}{5}\langle \phi_1|\phi_2\rangle\right)^2 \\
    P(\phi_1|\psi_1) &= \left(\frac{3}{5}\overbrace{\langle \phi_1|
      \phi_1 \rangle}^1
    + \frac{4}{5}\cancel{\langle \phi_1|\phi_2\rangle}\right)^2 \\
    P(\phi_1|\psi_1) &= \frac{9}{25} \\
  \end{align*}

  The probability of getting state $\phi_1$ given state $\psi_2$:

  \begin{align*}
    P(\phi_1|\psi_2) &= |\langle \phi_1|\psi_2 \rangle|^2 =
    \left(\frac{4}{5}\overbrace{|\langle \phi_1| \phi_1 \rangle}^1
    - \frac{3}{5}\cancel{|\langle \phi_1|\phi_2\rangle}\right)^2 \\
    P(\phi_1|\psi_2) &= \frac{16}{25}
  \end{align*}

  Total probability:

  \begin{align*}
    P(\phi_2) &= P(\phi_1|\psi_1)P(\psi_1) + P(\phi_1|\psi_2)P(\psi_2) \\
    P(\phi_2) &= \left(\frac{9}{25}\right)^2 + \left(\frac{16}{25}\right)^2 \\
    \Aboxed{P(\phi_2) &= \frac{337}{625}}
  \end{align*}
}

\pagebreak

\section*{4.10.3}

Consider the orthonormal basis functions $\psi_1(x) = 1/\sqrt2$ and
$\psi_2(x) = \sqrt{\frac{3}{2}}x$
that are capable of representing any function of the form $f(x) = ax
+ b$ defined over the range $-1 < x < 1$.

\begin{enumerate}[(i)]
  \item Consider now the new basis functions $\phi_1(x) =
    \frac{\sqrt{3}x}{2} + \frac{1}{2}$ and
    $\phi_2(x) = \frac{\sqrt{3}x}{2} - \frac{1}{2}$.
    Represent the functions $\phi_1(x)$ and $\phi_2(x)$ in a
    two-dimensional diagram with orthogonal axes
    corresponding to the functions $\psi_1(x)$ and $\psi_2(x)$
    respectively. \label{first}

    \boxedanswer{

      Express $f(x)$ in terms of $\psi_1$ and $\psi_2$:

      \begin{align*}
        1 &= \overbrace{\sqrt2}^b\psi_1(x) \\
        x &= \overbrace{\sqrt\frac{2}{3}}^a\psi_2(x)
      \end{align*}

      In the $[\psi_1, \psi_2]$ space, $f(x) = b\cdot 1 + a\cdot x$
      can be represented as $\left[b\sqrt{2}, a\sqrt{\frac{2}{3}}\right]$.

      For $\phi_1$ and $\phi_2$:

      \begin{align*}
        \phi_1(x) &= \overbrace{\frac{\sqrt{3}}{2}}^{a}x +
        \overbrace{\frac{1}{2}}^{b} \\
        \phi_2(x) &=
        \overbrace{\frac{\sqrt{3}}{2}}^{a}x
        \overbrace{-\frac{1}{2}}^{b}
      \end{align*}

      $\phi_1:$ $\left[\frac{\sqrt{2}}{2}, \frac{\sqrt{2}}{2}\right]$

      $\phi_2:$ $\left[-\frac{\sqrt{2}}{2}, \frac{\sqrt{2}}{2}\right]$

      \centering

      \input{src/fig_4_10_3.tex}

    }

    \pagebreak

  \item Construct the matrix that will transform a function in the
    “old” representation as a vector $
    \begin{bmatrix} c_1 \\ c_2
    \end{bmatrix}$
    into a new representation in terms of these new basis functions
    as a vector $
    \begin{bmatrix}d_1 \\ d_2
    \end{bmatrix}$, where an arbitrary function
    $f(x) = ax + b$
    is represented as the linear combination
    $f(x) = d_1 \phi_1(x) + d_2\phi_2(x)$. \label{second}
  \item Show that the matrix from part \ref{second} is unitary.
  \item Use the matrix of part \ref{second} to calculate the vector
    $
    \begin{bmatrix}d_1 \\ d_2
    \end{bmatrix}$
    for the specific example function $2x + 3$.
  \item Indicate the resulting vector on the same diagram as used for
    parts \ref{first}.
\end{enumerate}

\pagebreak

\section*{4.11.1}

For each of the following matrices, say whether or not it is unitary
and whether or not it is Hermitian.

\begin{enumerate}[(i)]
  \item $
    \begin{bmatrix} 1 & 0 \\ 0 & 1
    \end{bmatrix}$
  \item $
    \begin{bmatrix} 1 & i \\ -i & 1
    \end{bmatrix}$
  \item $
    \begin{bmatrix} i & 0 \\ 0 & i
    \end{bmatrix}$
  \item $
    \begin{bmatrix} 0 & 1 \\ i & 0
    \end{bmatrix}$
\end{enumerate}

\pagebreak

\section*{4.11.3 (i)}

Consider the Hermiticity of the following operators.

\begin{enumerate}[(i)]
  \item Prove that the momentum operator is Hermitian. For simplicity
    you may perform this proof
    for a one-dimensional system (i.e., only consider functions of
    $x$, and consider only the $\hat p_x$ operator).

    [Hints: Consider $\int_{-\infty}^\infty \psi_i^*(x)p_x\psi_j(x)dx$
      where the $\psi_n(x)$ are a complete orthonormal set.
      You may want to consider an integration by parts. Note that the
      $\psi_n(x)$ must vanish at $\pm \infty$,
    since otherwise they could not be normalized.]
\end{enumerate}

\pagebreak

\section*{5.1.1}

The Pauli spin matrices are quantum mechanical operators that operate
in a two-dimensional
Hilbert space, and can be written as

\begin{equation*}
  \hat\sigma_x =
  \begin{bmatrix} 0 & 1 \\ 1 & 0
  \end{bmatrix}, \hat\sigma_y =
  \begin{bmatrix} 0 & -i \\ i & 0
  \end{bmatrix}, \hat\sigma_z =
  \begin{bmatrix}
    1 & 0 \\ 0 & -1
  \end{bmatrix}
\end{equation*}

Find the commutation relations between each pair of these operators,
proving your answer by explicit
matrix multiplication, and simplifying the answers as much as possible.

\pagebreak

\section*{5.4.3}

Formally transform the momentum operator $\hat p_z$ into the momentum
basis using algebra similar to
that above for the transformation of the position operator into the
momentum basis.

\pagebreak

\end{document}

% Additional SI unit for Fahrenheit
\DeclareSIUnit\fahrenheit{\degree F}

\begin{document}

% Include title page
\input{./src/titlepage.tex}

\pagebreak

\section*{4.1.2}

Suppose that there are two quantum-mechanically measurable
quantities, $c$ with associated operator
$\hat C$ , and $d$ with associated operator $\hat D$ . In particular,
operator $\hat C$ has two eigenvectors $|\phi_1\rangle$ and $|\phi_2\rangle$,
and similarly operator $\hat D$ has two eigenvectors $|\psi_1\rangle$
and $|\psi_2\rangle$. The relation between the
eigenvectors is

\begin{align*}
  |\phi_1\rangle &= \frac{1}{5}(3|\psi_1\rangle + 4|\psi_2\rangle) \\
  |\phi_2\rangle &= \frac{1}{5}(4|\psi_1\rangle - 3|\psi_2\rangle)
\end{align*}

Suppose a measurement is made of the quantity $c$, and the system is
measured to be in state $|\phi_1\rangle$.
Then a measurement is made of quantity $d$, and following that the
quantity $c$ is again measured. What
is the probability (expressed as a fraction) that the system will be
found in state $|\psi_1\rangle$ on this second
measurement of $c$?

[Note: this is really a problem in quantum
  mechanical measurement discussed in
the previous Chapter, but is a good exercise in the use of the Dirac notation.]

\boxedanswer{

  \begin{align*}
    \hat C |\phi_1\rangle &= c|\phi_1\rangle \\
    \hat C |\phi_2\rangle &= c|\phi_2\rangle \\
    \hat D |\psi_1\rangle &= d|\phi_1\rangle \\
    \hat D |\psi_2\rangle &= d|\phi_2\rangle \\
    | \phi_1 \rangle &= \frac{1}{5}(3|\psi_1\rangle + 4 | \psi_2 \rangle) \\
    |\psi_1\rangle &= \frac{5}{3}| \phi_1 \rangle - \frac{4}{3}|
    \psi_2 \rangle \\
    |\phi_2\rangle &= \frac{1}{5}(4|\psi_1\rangle - 3|\psi_2\rangle) \\
    |\psi_2\rangle &= \frac{4}{3}|\psi_1\rangle - \frac{5}{3}|\phi_2\rangle \\
    |\psi_1\rangle &= \frac{5}{3}| \phi_1 \rangle - \frac{4}{3}
    \left(\frac{4}{3}|\psi_1\rangle -
    \frac{5}{3}|\phi_2\rangle\right) \\
    |\psi_1\rangle &= \frac{5}{3}| \phi_1 \rangle
    - \frac{16}{9}|\psi_1\rangle
    + \frac{20}{9}|\phi_2\rangle \\
    \frac{25}{9}|\psi_1\rangle
    &= \frac{5}{3}| \phi_1 \rangle
    + \frac{20}{9}|\phi_2\rangle \\
    |\psi_1\rangle
    &= \frac{3}{5}| \phi_1 \rangle
    + \frac{4}{5}|\phi_2\rangle \\
    |\psi_2\rangle &= \frac{4}{3}\left( \frac{3}{5}| \phi_1 \rangle
    + \frac{4}{5}|\phi_2\rangle\right) - \frac{5}{3}|\phi_2\rangle \\
    |\psi_2\rangle &= \frac{4}{5}| \phi_1 \rangle
    - \frac{3}{5}|\phi_2\rangle \\
  \end{align*}

  The probability of getting state $\psi_1$ after measuring $d$,
  given that the system is in state $\phi_1$.:

  \begin{align*}
    P(\psi_1) &= |\langle\psi_1|\phi_1\rangle|^2 =
    \left(\frac{3}{5}\overbrace{\langle\psi_1|\psi_1\rangle}^1 + \frac{4}{5}
    \cancel{\langle\psi_1| \psi_2 \rangle}\right)^2 \\
    P(\psi_1) &= \frac{9}{25}
  \end{align*}

}

\boxedanswer{
  The probability of getting state $\psi_2$ after measuring $d$,
  given that the system is in state $\phi_1$.:

  \begin{align*}
    P(\psi_2) &= |\langle\psi_2|\phi_1\rangle|^2 =
    \left(\frac{3}{5}\cancel{\langle\psi_2|\psi_1\rangle} + \frac{4}{5}
    \overbrace{\langle\psi_2| \psi_2 \rangle}^1\right)^2 \\
    P(\psi_2) &= \frac{16}{25}
  \end{align*}

  The probability of getting state $\phi_1$ given state $\psi_1$:

  \begin{align*}
    P(\phi_1|\psi_1) &= |\langle \phi_1|\psi_1 \rangle|^2 =
    \left(\frac{3}{5}\langle \phi_1| \phi_1 \rangle
    + \frac{4}{5}\langle \phi_1|\phi_2\rangle\right)^2 \\
    P(\phi_1|\psi_1) &= \left(\frac{3}{5}\overbrace{\langle \phi_1|
      \phi_1 \rangle}^1
    + \frac{4}{5}\cancel{\langle \phi_1|\phi_2\rangle}\right)^2 \\
    P(\phi_1|\psi_1) &= \frac{9}{25} \\
  \end{align*}

  The probability of getting state $\phi_1$ given state $\psi_2$:

  \begin{align*}
    P(\phi_1|\psi_2) &= |\langle \phi_1|\psi_2 \rangle|^2 =
    \left(\frac{4}{5}\overbrace{|\langle \phi_1| \phi_1 \rangle}^1
    - \frac{3}{5}\cancel{|\langle \phi_1|\phi_2\rangle}\right)^2 \\
    P(\phi_1|\psi_2) &= \frac{16}{25}
  \end{align*}

  Total probability:

  \begin{align*}
    P(\phi_2) &= P(\phi_1|\psi_1)P(\psi_1) + P(\phi_1|\psi_2)P(\psi_2) \\
    P(\phi_2) &= \left(\frac{9}{25}\right)^2 + \left(\frac{16}{25}\right)^2 \\
    \Aboxed{P(\phi_2) &= \frac{337}{625}}
  \end{align*}
}

\pagebreak

\section*{4.10.3}

Consider the orthonormal basis functions $\psi_1(x) = 1/\sqrt2$ and
$\psi_2(x) = \sqrt{\frac{3}{2}}x$
that are capable of representing any function of the form $f(x) = ax
+ b$ defined over the range $-1 < x < 1$.

\begin{enumerate}[(i)]
  \item Consider now the new basis functions $\phi_1(x) =
    \frac{\sqrt{3}x}{2} + \frac{1}{2}$ and
    $\phi_2(x) = \frac{\sqrt{3}x}{2} - \frac{1}{2}$.
    Represent the functions $\phi_1(x)$ and $\phi_2(x)$ in a
    two-dimensional diagram with orthogonal axes
    corresponding to the functions $\psi_1(x)$ and $\psi_2(x)$
    respectively. \label{first}

    \boxedanswer{

      Express $f(x)$ in terms of $\psi_1$ and $\psi_2$:

      \begin{align*}
        1 &= \overbrace{\sqrt2}^b\psi_1(x) \\
        x &= \overbrace{\sqrt\frac{2}{3}}^a\psi_2(x)
      \end{align*}

      In the $[\psi_1, \psi_2]$ space, $f(x) = b\cdot 1 + a\cdot x$
      can be represented as $\left[b\sqrt{2}, a\sqrt{\frac{2}{3}}\right]$.

      For $\phi_1$ and $\phi_2$:

      \begin{align*}
        \phi_1(x) &= \overbrace{\frac{\sqrt{3}}{2}}^{a}x +
        \overbrace{\frac{1}{2}}^{b} \\
        \phi_2(x) &=
        \overbrace{\frac{\sqrt{3}}{2}}^{a}x
        \overbrace{-\frac{1}{2}}^{b}
      \end{align*}

      $\phi_1:$ $\left[\frac{\sqrt{2}}{2}, \frac{\sqrt{2}}{2}\right]$

      $\phi_2:$ $\left[-\frac{\sqrt{2}}{2}, \frac{\sqrt{2}}{2}\right]$

      \centering

      \input{src/fig_4_10_3.tex}

    }

    \pagebreak

  \item Construct the matrix that will transform a function in the
    “old” representation as a vector $
    \begin{bmatrix} c_1 \\ c_2
    \end{bmatrix}$
    into a new representation in terms of these new basis functions
    as a vector $
    \begin{bmatrix}d_1 \\ d_2
    \end{bmatrix}$, where an arbitrary function
    $f(x) = ax + b$
    is represented as the linear combination
    $f(x) = d_1 \phi_1(x) + d_2\phi_2(x)$. \label{second}
  \item Show that the matrix from part \ref{second} is unitary.
  \item Use the matrix of part \ref{second} to calculate the vector
    $
    \begin{bmatrix}d_1 \\ d_2
    \end{bmatrix}$
    for the specific example function $2x + 3$.
  \item Indicate the resulting vector on the same diagram as used for
    parts \ref{first}.
\end{enumerate}

\pagebreak

\section*{4.11.1}

For each of the following matrices, say whether or not it is unitary
and whether or not it is Hermitian.

\begin{enumerate}[(i)]
  \item $
    \begin{bmatrix} 1 & 0 \\ 0 & 1
    \end{bmatrix}$
  \item $
    \begin{bmatrix} 1 & i \\ -i & 1
    \end{bmatrix}$
  \item $
    \begin{bmatrix} i & 0 \\ 0 & i
    \end{bmatrix}$
  \item $
    \begin{bmatrix} 0 & 1 \\ i & 0
    \end{bmatrix}$
\end{enumerate}

\pagebreak

\section*{4.11.3 (i)}

Consider the Hermiticity of the following operators.

\begin{enumerate}[(i)]
  \item Prove that the momentum operator is Hermitian. For simplicity
    you may perform this proof
    for a one-dimensional system (i.e., only consider functions of
    $x$, and consider only the $\hat p_x$ operator).

    [Hints: Consider $\int_{-\infty}^\infty \psi_i^*(x)p_x\psi_j(x)dx$
      where the $\psi_n(x)$ are a complete orthonormal set.
      You may want to consider an integration by parts. Note that the
      $\psi_n(x)$ must vanish at $\pm \infty$,
    since otherwise they could not be normalized.]
\end{enumerate}

\pagebreak

\section*{5.1.1}

The Pauli spin matrices are quantum mechanical operators that operate
in a two-dimensional
Hilbert space, and can be written as

\begin{equation*}
  \hat\sigma_x =
  \begin{bmatrix} 0 & 1 \\ 1 & 0
  \end{bmatrix}, \hat\sigma_y =
  \begin{bmatrix} 0 & -i \\ i & 0
  \end{bmatrix}, \hat\sigma_z =
  \begin{bmatrix}
    1 & 0 \\ 0 & -1
  \end{bmatrix}
\end{equation*}

Find the commutation relations between each pair of these operators,
proving your answer by explicit
matrix multiplication, and simplifying the answers as much as possible.

\pagebreak

\section*{5.4.3}

Formally transform the momentum operator $\hat p_z$ into the momentum
basis using algebra similar to
that above for the transformation of the position operator into the
momentum basis.

\pagebreak

\end{document}

% Additional SI unit for Fahrenheit
\DeclareSIUnit\fahrenheit{\degree F}

\begin{document}

% Include title page
\input{./src/titlepage.tex}

\pagebreak

\section*{4.1.2}

Suppose that there are two quantum-mechanically measurable
quantities, $c$ with associated operator
$\hat C$ , and $d$ with associated operator $\hat D$ . In particular,
operator $\hat C$ has two eigenvectors $|\phi_1\rangle$ and $|\phi_2\rangle$,
and similarly operator $\hat D$ has two eigenvectors $|\psi_1\rangle$
and $|\psi_2\rangle$. The relation between the
eigenvectors is

\begin{align*}
  |\phi_1\rangle &= \frac{1}{5}(3|\psi_1\rangle + 4|\psi_2\rangle) \\
  |\phi_2\rangle &= \frac{1}{5}(4|\psi_1\rangle - 3|\psi_2\rangle)
\end{align*}

Suppose a measurement is made of the quantity $c$, and the system is
measured to be in state $|\phi_1\rangle$.
Then a measurement is made of quantity $d$, and following that the
quantity $c$ is again measured. What
is the probability (expressed as a fraction) that the system will be
found in state $|\psi_1\rangle$ on this second
measurement of $c$?

[Note: this is really a problem in quantum
  mechanical measurement discussed in
the previous Chapter, but is a good exercise in the use of the Dirac notation.]

\boxedanswer{

  \begin{align*}
    \hat C |\phi_1\rangle &= c|\phi_1\rangle \\
    \hat C |\phi_2\rangle &= c|\phi_2\rangle \\
    \hat D |\psi_1\rangle &= d|\phi_1\rangle \\
    \hat D |\psi_2\rangle &= d|\phi_2\rangle \\
    | \phi_1 \rangle &= \frac{1}{5}(3|\psi_1\rangle + 4 | \psi_2 \rangle) \\
    |\psi_1\rangle &= \frac{5}{3}| \phi_1 \rangle - \frac{4}{3}|
    \psi_2 \rangle \\
    |\phi_2\rangle &= \frac{1}{5}(4|\psi_1\rangle - 3|\psi_2\rangle) \\
    |\psi_2\rangle &= \frac{4}{3}|\psi_1\rangle - \frac{5}{3}|\phi_2\rangle \\
    |\psi_1\rangle &= \frac{5}{3}| \phi_1 \rangle - \frac{4}{3}
    \left(\frac{4}{3}|\psi_1\rangle -
    \frac{5}{3}|\phi_2\rangle\right) \\
    |\psi_1\rangle &= \frac{5}{3}| \phi_1 \rangle
    - \frac{16}{9}|\psi_1\rangle
    + \frac{20}{9}|\phi_2\rangle \\
    \frac{25}{9}|\psi_1\rangle
    &= \frac{5}{3}| \phi_1 \rangle
    + \frac{20}{9}|\phi_2\rangle \\
    |\psi_1\rangle
    &= \frac{3}{5}| \phi_1 \rangle
    + \frac{4}{5}|\phi_2\rangle \\
    |\psi_2\rangle &= \frac{4}{3}\left( \frac{3}{5}| \phi_1 \rangle
    + \frac{4}{5}|\phi_2\rangle\right) - \frac{5}{3}|\phi_2\rangle \\
    |\psi_2\rangle &= \frac{4}{5}| \phi_1 \rangle
    - \frac{3}{5}|\phi_2\rangle \\
  \end{align*}

  The probability of getting state $\psi_1$ after measuring $d$,
  given that the system is in state $\phi_1$.:

  \begin{align*}
    P(\psi_1) &= |\langle\psi_1|\phi_1\rangle|^2 =
    \left(\frac{3}{5}\overbrace{\langle\psi_1|\psi_1\rangle}^1 + \frac{4}{5}
    \cancel{\langle\psi_1| \psi_2 \rangle}\right)^2 \\
    P(\psi_1) &= \frac{9}{25}
  \end{align*}

}

\boxedanswer{
  The probability of getting state $\psi_2$ after measuring $d$,
  given that the system is in state $\phi_1$.:

  \begin{align*}
    P(\psi_2) &= |\langle\psi_2|\phi_1\rangle|^2 =
    \left(\frac{3}{5}\cancel{\langle\psi_2|\psi_1\rangle} + \frac{4}{5}
    \overbrace{\langle\psi_2| \psi_2 \rangle}^1\right)^2 \\
    P(\psi_2) &= \frac{16}{25}
  \end{align*}

  The probability of getting state $\phi_1$ given state $\psi_1$:

  \begin{align*}
    P(\phi_1|\psi_1) &= |\langle \phi_1|\psi_1 \rangle|^2 =
    \left(\frac{3}{5}\langle \phi_1| \phi_1 \rangle
    + \frac{4}{5}\langle \phi_1|\phi_2\rangle\right)^2 \\
    P(\phi_1|\psi_1) &= \left(\frac{3}{5}\overbrace{\langle \phi_1|
      \phi_1 \rangle}^1
    + \frac{4}{5}\cancel{\langle \phi_1|\phi_2\rangle}\right)^2 \\
    P(\phi_1|\psi_1) &= \frac{9}{25} \\
  \end{align*}

  The probability of getting state $\phi_1$ given state $\psi_2$:

  \begin{align*}
    P(\phi_1|\psi_2) &= |\langle \phi_1|\psi_2 \rangle|^2 =
    \left(\frac{4}{5}\overbrace{|\langle \phi_1| \phi_1 \rangle}^1
    - \frac{3}{5}\cancel{|\langle \phi_1|\phi_2\rangle}\right)^2 \\
    P(\phi_1|\psi_2) &= \frac{16}{25}
  \end{align*}

  Total probability:

  \begin{align*}
    P(\phi_2) &= P(\phi_1|\psi_1)P(\psi_1) + P(\phi_1|\psi_2)P(\psi_2) \\
    P(\phi_2) &= \left(\frac{9}{25}\right)^2 + \left(\frac{16}{25}\right)^2 \\
    \Aboxed{P(\phi_2) &= \frac{337}{625}}
  \end{align*}
}

\pagebreak

\section*{4.10.3}

Consider the orthonormal basis functions $\psi_1(x) = 1/\sqrt2$ and
$\psi_2(x) = \sqrt{\frac{3}{2}}x$
that are capable of representing any function of the form $f(x) = ax
+ b$ defined over the range $-1 < x < 1$.

\begin{enumerate}[(i)]
  \item Consider now the new basis functions $\phi_1(x) =
    \frac{\sqrt{3}x}{2} + \frac{1}{2}$ and
    $\phi_2(x) = \frac{\sqrt{3}x}{2} - \frac{1}{2}$.
    Represent the functions $\phi_1(x)$ and $\phi_2(x)$ in a
    two-dimensional diagram with orthogonal axes
    corresponding to the functions $\psi_1(x)$ and $\psi_2(x)$
    respectively. \label{first}

    \boxedanswer{

      Express $f(x)$ in terms of $\psi_1$ and $\psi_2$:

      \begin{align*}
        1 &= \overbrace{\sqrt2}^b\psi_1(x) \\
        x &= \overbrace{\sqrt\frac{2}{3}}^a\psi_2(x)
      \end{align*}

      In the $[\psi_1, \psi_2]$ space, $f(x) = b\cdot 1 + a\cdot x$
      can be represented as $\left[b\sqrt{2}, a\sqrt{\frac{2}{3}}\right]$.

      For $\phi_1$ and $\phi_2$:

      \begin{align*}
        \phi_1(x) &= \overbrace{\frac{\sqrt{3}}{2}}^{a}x +
        \overbrace{\frac{1}{2}}^{b} \\
        \phi_2(x) &=
        \overbrace{\frac{\sqrt{3}}{2}}^{a}x
        \overbrace{-\frac{1}{2}}^{b}
      \end{align*}

      $\phi_1:$ $\left[\frac{\sqrt{2}}{2}, \frac{\sqrt{2}}{2}\right]$

      $\phi_2:$ $\left[-\frac{\sqrt{2}}{2}, \frac{\sqrt{2}}{2}\right]$

      \centering

      \input{src/fig_4_10_3.tex}

    }

    \pagebreak

  \item Construct the matrix that will transform a function in the
    “old” representation as a vector $
    \begin{bmatrix} c_1 \\ c_2
    \end{bmatrix}$
    into a new representation in terms of these new basis functions
    as a vector $
    \begin{bmatrix}d_1 \\ d_2
    \end{bmatrix}$, where an arbitrary function
    $f(x) = ax + b$
    is represented as the linear combination
    $f(x) = d_1 \phi_1(x) + d_2\phi_2(x)$. \label{second}
  \item Show that the matrix from part \ref{second} is unitary.
  \item Use the matrix of part \ref{second} to calculate the vector
    $
    \begin{bmatrix}d_1 \\ d_2
    \end{bmatrix}$
    for the specific example function $2x + 3$.
  \item Indicate the resulting vector on the same diagram as used for
    parts \ref{first}.
\end{enumerate}

\pagebreak

\section*{4.11.1}

For each of the following matrices, say whether or not it is unitary
and whether or not it is Hermitian.

\begin{enumerate}[(i)]
  \item $
    \begin{bmatrix} 1 & 0 \\ 0 & 1
    \end{bmatrix}$
  \item $
    \begin{bmatrix} 1 & i \\ -i & 1
    \end{bmatrix}$
  \item $
    \begin{bmatrix} i & 0 \\ 0 & i
    \end{bmatrix}$
  \item $
    \begin{bmatrix} 0 & 1 \\ i & 0
    \end{bmatrix}$
\end{enumerate}

\pagebreak

\section*{4.11.3 (i)}

Consider the Hermiticity of the following operators.

\begin{enumerate}[(i)]
  \item Prove that the momentum operator is Hermitian. For simplicity
    you may perform this proof
    for a one-dimensional system (i.e., only consider functions of
    $x$, and consider only the $\hat p_x$ operator).

    [Hints: Consider $\int_{-\infty}^\infty \psi_i^*(x)p_x\psi_j(x)dx$
      where the $\psi_n(x)$ are a complete orthonormal set.
      You may want to consider an integration by parts. Note that the
      $\psi_n(x)$ must vanish at $\pm \infty$,
    since otherwise they could not be normalized.]
\end{enumerate}

\pagebreak

\section*{5.1.1}

The Pauli spin matrices are quantum mechanical operators that operate
in a two-dimensional
Hilbert space, and can be written as

\begin{equation*}
  \hat\sigma_x =
  \begin{bmatrix} 0 & 1 \\ 1 & 0
  \end{bmatrix}, \hat\sigma_y =
  \begin{bmatrix} 0 & -i \\ i & 0
  \end{bmatrix}, \hat\sigma_z =
  \begin{bmatrix}
    1 & 0 \\ 0 & -1
  \end{bmatrix}
\end{equation*}

Find the commutation relations between each pair of these operators,
proving your answer by explicit
matrix multiplication, and simplifying the answers as much as possible.

\pagebreak

\section*{5.4.3}

Formally transform the momentum operator $\hat p_z$ into the momentum
basis using algebra similar to
that above for the transformation of the position operator into the
momentum basis.

\pagebreak

\end{document}

% Additional SI unit for Fahrenheit
\DeclareSIUnit\fahrenheit{\degree F}

\begin{document}

% Include title page
\input{./src/titlepage.tex}

\pagebreak

\section*{4.1.2}

Suppose that there are two quantum-mechanically measurable
quantities, $c$ with associated operator
$\hat C$ , and $d$ with associated operator $\hat D$ . In particular,
operator $\hat C$ has two eigenvectors $|\phi_1\rangle$ and $|\phi_2\rangle$,
and similarly operator $\hat D$ has two eigenvectors $|\psi_1\rangle$
and $|\psi_2\rangle$. The relation between the
eigenvectors is

\begin{align*}
  |\phi_1\rangle &= \frac{1}{5}(3|\psi_1\rangle + 4|\psi_2\rangle) \\
  |\phi_2\rangle &= \frac{1}{5}(4|\psi_1\rangle - 3|\psi_2\rangle)
\end{align*}

Suppose a measurement is made of the quantity $c$, and the system is
measured to be in state $|\phi_1\rangle$.
Then a measurement is made of quantity $d$, and following that the
quantity $c$ is again measured. What
is the probability (expressed as a fraction) that the system will be
found in state $|\psi_1\rangle$ on this second
measurement of $c$?

[Note: this is really a problem in quantum
  mechanical measurement discussed in
the previous Chapter, but is a good exercise in the use of the Dirac notation.]

\boxedanswer{

  \begin{align*}
    \hat C |\phi_1\rangle &= c|\phi_1\rangle \\
    \hat C |\phi_2\rangle &= c|\phi_2\rangle \\
    \hat D |\psi_1\rangle &= d|\phi_1\rangle \\
    \hat D |\psi_2\rangle &= d|\phi_2\rangle \\
    | \phi_1 \rangle &= \frac{1}{5}(3|\psi_1\rangle + 4 | \psi_2 \rangle) \\
    |\psi_1\rangle &= \frac{5}{3}| \phi_1 \rangle - \frac{4}{3}|
    \psi_2 \rangle \\
    |\phi_2\rangle &= \frac{1}{5}(4|\psi_1\rangle - 3|\psi_2\rangle) \\
    |\psi_2\rangle &= \frac{4}{3}|\psi_1\rangle - \frac{5}{3}|\phi_2\rangle \\
    |\psi_1\rangle &= \frac{5}{3}| \phi_1 \rangle - \frac{4}{3}
    \left(\frac{4}{3}|\psi_1\rangle -
    \frac{5}{3}|\phi_2\rangle\right) \\
    |\psi_1\rangle &= \frac{5}{3}| \phi_1 \rangle
    - \frac{16}{9}|\psi_1\rangle
    + \frac{20}{9}|\phi_2\rangle \\
    \frac{25}{9}|\psi_1\rangle
    &= \frac{5}{3}| \phi_1 \rangle
    + \frac{20}{9}|\phi_2\rangle \\
    |\psi_1\rangle
    &= \frac{3}{5}| \phi_1 \rangle
    + \frac{4}{5}|\phi_2\rangle \\
    |\psi_2\rangle &= \frac{4}{3}\left( \frac{3}{5}| \phi_1 \rangle
    + \frac{4}{5}|\phi_2\rangle\right) - \frac{5}{3}|\phi_2\rangle \\
    |\psi_2\rangle &= \frac{4}{5}| \phi_1 \rangle
    - \frac{3}{5}|\phi_2\rangle \\
  \end{align*}

  The probability of getting state $\psi_1$ after measuring $d$,
  given that the system is in state $\phi_1$.:

  \begin{align*}
    P(\psi_1) &= |\langle\psi_1|\phi_1\rangle|^2 =
    \left(\frac{3}{5}\overbrace{\langle\psi_1|\psi_1\rangle}^1 + \frac{4}{5}
    \cancel{\langle\psi_1| \psi_2 \rangle}\right)^2 \\
    P(\psi_1) &= \frac{9}{25}
  \end{align*}

}

\boxedanswer{
  The probability of getting state $\psi_2$ after measuring $d$,
  given that the system is in state $\phi_1$.:

  \begin{align*}
    P(\psi_2) &= |\langle\psi_2|\phi_1\rangle|^2 =
    \left(\frac{3}{5}\cancel{\langle\psi_2|\psi_1\rangle} + \frac{4}{5}
    \overbrace{\langle\psi_2| \psi_2 \rangle}^1\right)^2 \\
    P(\psi_2) &= \frac{16}{25}
  \end{align*}

  The probability of getting state $\phi_1$ given state $\psi_1$:

  \begin{align*}
    P(\phi_1|\psi_1) &= |\langle \phi_1|\psi_1 \rangle|^2 =
    \left(\frac{3}{5}\langle \phi_1| \phi_1 \rangle
    + \frac{4}{5}\langle \phi_1|\phi_2\rangle\right)^2 \\
    P(\phi_1|\psi_1) &= \left(\frac{3}{5}\overbrace{\langle \phi_1|
      \phi_1 \rangle}^1
    + \frac{4}{5}\cancel{\langle \phi_1|\phi_2\rangle}\right)^2 \\
    P(\phi_1|\psi_1) &= \frac{9}{25} \\
  \end{align*}

  The probability of getting state $\phi_1$ given state $\psi_2$:

  \begin{align*}
    P(\phi_1|\psi_2) &= |\langle \phi_1|\psi_2 \rangle|^2 =
    \left(\frac{4}{5}\overbrace{|\langle \phi_1| \phi_1 \rangle}^1
    - \frac{3}{5}\cancel{|\langle \phi_1|\phi_2\rangle}\right)^2 \\
    P(\phi_1|\psi_2) &= \frac{16}{25}
  \end{align*}

  Total probability:

  \begin{align*}
    P(\phi_2) &= P(\phi_1|\psi_1)P(\psi_1) + P(\phi_1|\psi_2)P(\psi_2) \\
    P(\phi_2) &= \left(\frac{9}{25}\right)^2 + \left(\frac{16}{25}\right)^2 \\
    \Aboxed{P(\phi_2) &= \frac{337}{625}}
  \end{align*}
}

\pagebreak

\section*{4.10.3}

Consider the orthonormal basis functions $\psi_1(x) = 1/\sqrt2$ and
$\psi_2(x) = \sqrt{\frac{3}{2}}x$
that are capable of representing any function of the form $f(x) = ax
+ b$ defined over the range $-1 < x < 1$.

\begin{enumerate}[(i)]
  \item Consider now the new basis functions $\phi_1(x) =
    \frac{\sqrt{3}x}{2} + \frac{1}{2}$ and
    $\phi_2(x) = \frac{\sqrt{3}x}{2} - \frac{1}{2}$.
    Represent the functions $\phi_1(x)$ and $\phi_2(x)$ in a
    two-dimensional diagram with orthogonal axes
    corresponding to the functions $\psi_1(x)$ and $\psi_2(x)$
    respectively. \label{first}

    \boxedanswer{

      Express $f(x)$ in terms of $\psi_1$ and $\psi_2$:

      \begin{align*}
        1 &= \overbrace{\sqrt2}^b\psi_1(x) \\
        x &= \overbrace{\sqrt\frac{2}{3}}^a\psi_2(x)
      \end{align*}

      In the $[\psi_1, \psi_2]$ space, $f(x) = b\cdot 1 + a\cdot x$
      can be represented as $\left[b\sqrt{2}, a\sqrt{\frac{2}{3}}\right]$.

      For $\phi_1$ and $\phi_2$:

      \begin{align*}
        \phi_1(x) &= \overbrace{\frac{\sqrt{3}}{2}}^{a}x +
        \overbrace{\frac{1}{2}}^{b} \\
        \phi_2(x) &=
        \overbrace{\frac{\sqrt{3}}{2}}^{a}x
        \overbrace{-\frac{1}{2}}^{b}
      \end{align*}

      $\phi_1:$ $\left[\frac{\sqrt{2}}{2}, \frac{\sqrt{2}}{2}\right]$

      $\phi_2:$ $\left[-\frac{\sqrt{2}}{2}, \frac{\sqrt{2}}{2}\right]$

      \centering

      \input{src/fig_4_10_3.tex}

    }

    \pagebreak

  \item Construct the matrix that will transform a function in the
    “old” representation as a vector $
    \begin{bmatrix} c_1 \\ c_2
    \end{bmatrix}$
    into a new representation in terms of these new basis functions
    as a vector $
    \begin{bmatrix}d_1 \\ d_2
    \end{bmatrix}$, where an arbitrary function
    $f(x) = ax + b$
    is represented as the linear combination
    $f(x) = d_1 \phi_1(x) + d_2\phi_2(x)$. \label{second}
  \item Show that the matrix from part \ref{second} is unitary.
  \item Use the matrix of part \ref{second} to calculate the vector
    $
    \begin{bmatrix}d_1 \\ d_2
    \end{bmatrix}$
    for the specific example function $2x + 3$.
  \item Indicate the resulting vector on the same diagram as used for
    parts \ref{first}.
\end{enumerate}

\pagebreak

\section*{4.11.1}

For each of the following matrices, say whether or not it is unitary
and whether or not it is Hermitian.

\begin{enumerate}[(i)]
  \item $
    \begin{bmatrix} 1 & 0 \\ 0 & 1
    \end{bmatrix}$
  \item $
    \begin{bmatrix} 1 & i \\ -i & 1
    \end{bmatrix}$
  \item $
    \begin{bmatrix} i & 0 \\ 0 & i
    \end{bmatrix}$
  \item $
    \begin{bmatrix} 0 & 1 \\ i & 0
    \end{bmatrix}$
\end{enumerate}

\pagebreak

\section*{4.11.3 (i)}

Consider the Hermiticity of the following operators.

\begin{enumerate}[(i)]
  \item Prove that the momentum operator is Hermitian. For simplicity
    you may perform this proof
    for a one-dimensional system (i.e., only consider functions of
    $x$, and consider only the $\hat p_x$ operator).

    [Hints: Consider $\int_{-\infty}^\infty \psi_i^*(x)p_x\psi_j(x)dx$
      where the $\psi_n(x)$ are a complete orthonormal set.
      You may want to consider an integration by parts. Note that the
      $\psi_n(x)$ must vanish at $\pm \infty$,
    since otherwise they could not be normalized.]
\end{enumerate}

\pagebreak

\section*{5.1.1}

The Pauli spin matrices are quantum mechanical operators that operate
in a two-dimensional
Hilbert space, and can be written as

\begin{equation*}
  \hat\sigma_x =
  \begin{bmatrix} 0 & 1 \\ 1 & 0
  \end{bmatrix}, \hat\sigma_y =
  \begin{bmatrix} 0 & -i \\ i & 0
  \end{bmatrix}, \hat\sigma_z =
  \begin{bmatrix}
    1 & 0 \\ 0 & -1
  \end{bmatrix}
\end{equation*}

Find the commutation relations between each pair of these operators,
proving your answer by explicit
matrix multiplication, and simplifying the answers as much as possible.

\pagebreak

\section*{5.4.3}

Formally transform the momentum operator $\hat p_z$ into the momentum
basis using algebra similar to
that above for the transformation of the position operator into the
momentum basis.

\pagebreak

\end{document}
