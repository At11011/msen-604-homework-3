\input{./src/main.sty}
% Additional SI unit for Fahrenheit
\DeclareSIUnit\fahrenheit{\degree F}

\begin{document}

% Include title page
\input{./src/titlepage.tex}

\pagebreak

\section*{4.1.2}

Suppose that there are two quantum-mechanically measurable
quantities, $c$ with associated operator
$\hat C$ , and $d$ with associated operator $\hat D$ . In particular,
operator $\hat C$ has two eigenvectors $|\phi_1\rangle$ and $|\phi_2\rangle$,
and similarly operator $\hat D$ has two eigenvectors $|\psi_1\rangle$
and $|\psi_2\rangle$. The relation between the
eigenvectors is

\begin{align*}
  |\phi_1\rangle &= \frac{1}{5}(3|\psi_1\rangle + 4|\psi_2\rangle) \\
  |\phi_2\rangle &= \frac{1}{5}(4|\psi_1\rangle - 3|\psi_2\rangle)
\end{align*}

Suppose a measurement is made of the quantity $c$, and the system is
measured to be in state $|\phi_1\rangle$.
Then a measurement is made of quantity $d$, and following that the
quantity $c$ is again measured. What
is the probability (expressed as a fraction) that the system will be
found in state $|\psi_1\rangle$ on this second
measurement of $c$?

[Note: this is really a problem in quantum
  mechanical measurement discussed in
the previous Chapter, but is a good exercise in the use of the Dirac notation.]

\boxedanswer{

  \begin{align*}
    \hat C |\phi_1\rangle &= c|\phi_1\rangle \\
    \hat C |\phi_2\rangle &= c|\phi_2\rangle \\
    \hat D |\psi_1\rangle &= d|\phi_1\rangle \\
    \hat D |\psi_2\rangle &= d|\phi_2\rangle \\
    | \phi_1 \rangle &= \frac{1}{5}(3|\psi_1\rangle + 4 | \psi_2 \rangle) \\
    |\psi_1\rangle &= \frac{5}{3}| \phi_1 \rangle - \frac{4}{3}|
    \psi_2 \rangle \\
    |\phi_2\rangle &= \frac{1}{5}(4|\psi_1\rangle - 3|\psi_2\rangle) \\
    |\psi_2\rangle &= \frac{4}{3}|\psi_1\rangle - \frac{5}{3}|\phi_2\rangle \\
    |\psi_1\rangle &= \frac{5}{3}| \phi_1 \rangle - \frac{4}{3}
    \left(\frac{4}{3}|\psi_1\rangle -
    \frac{5}{3}|\phi_2\rangle\right) \\
    |\psi_1\rangle &= \frac{5}{3}| \phi_1 \rangle
    - \frac{16}{9}|\psi_1\rangle
    + \frac{20}{9}|\phi_2\rangle \\
    \frac{25}{9}|\psi_1\rangle
    &= \frac{5}{3}| \phi_1 \rangle
    + \frac{20}{9}|\phi_2\rangle \\
    |\psi_1\rangle
    &= \frac{3}{5}| \phi_1 \rangle
    + \frac{4}{5}|\phi_2\rangle \\
    |\psi_2\rangle &= \frac{4}{3}\left( \frac{3}{5}| \phi_1 \rangle
    + \frac{4}{5}|\phi_2\rangle\right) - \frac{5}{3}|\phi_2\rangle \\
    |\psi_2\rangle &= \frac{4}{5}| \phi_1 \rangle
    - \frac{3}{5}|\phi_2\rangle \\
  \end{align*}

  The probability of getting state $\psi_1$ after measuring $d$,
  given that the system is in state $\phi_1$.:

  \begin{align*}
    P(\psi_1) &= |\langle\psi_1|\phi_1\rangle|^2 =
    \left(\frac{3}{5}\overbrace{\langle\psi_1|\psi_1\rangle}^1 + \frac{4}{5}
    \cancel{\langle\psi_1| \psi_2 \rangle}\right)^2 \\
    P(\psi_1) &= \frac{9}{25}
  \end{align*}

}

\boxedanswer{
  The probability of getting state $\psi_2$ after measuring $d$,
  given that the system is in state $\phi_1$.:

  \begin{align*}
    P(\psi_2) &= |\langle\psi_2|\phi_1\rangle|^2 =
    \left(\frac{3}{5}\cancel{\langle\psi_2|\psi_1\rangle} + \frac{4}{5}
    \overbrace{\langle\psi_2| \psi_2 \rangle}^1\right)^2 \\
    P(\psi_2) &= \frac{16}{25}
  \end{align*}

  The probability of getting state $\phi_1$ given state $\psi_1$:

  \begin{align*}
    P(\phi_1|\psi_1) &= |\langle \phi_1|\psi_1 \rangle|^2 =
    \left(\frac{3}{5}\langle \phi_1| \phi_1 \rangle
    + \frac{4}{5}\langle \phi_1|\phi_2\rangle\right)^2 \\
    P(\phi_1|\psi_1) &= \left(\frac{3}{5}\overbrace{\langle \phi_1|
      \phi_1 \rangle}^1
    + \frac{4}{5}\cancel{\langle \phi_1|\phi_2\rangle}\right)^2 \\
    P(\phi_1|\psi_1) &= \frac{9}{25} \\
  \end{align*}

  The probability of getting state $\phi_1$ given state $\psi_2$:

  \begin{align*}
    P(\phi_1|\psi_2) &= |\langle \phi_1|\psi_2 \rangle|^2 =
    \left(\frac{4}{5}\overbrace{|\langle \phi_1| \phi_1 \rangle}^1
    - \frac{3}{5}\cancel{|\langle \phi_1|\phi_2\rangle}\right)^2 \\
    P(\phi_1|\psi_2) &= \frac{16}{25}
  \end{align*}

  Total probability:

  \begin{align*}
    P(\phi_2) &= P(\phi_1|\psi_1)P(\psi_1) + P(\phi_1|\psi_2)P(\psi_2) \\
    P(\phi_2) &= \left(\frac{9}{25}\right)^2 + \left(\frac{16}{25}\right)^2 \\
    \Aboxed{P(\phi_2) &= \frac{337}{625}}
  \end{align*}
}

\pagebreak

\section*{4.10.3}

Consider the orthonormal basis functions $\psi_1(x) = 1/\sqrt2$ and
$\psi_2(x) = \sqrt{\frac{3}{2}}x$
that are capable of representing any function of the form $f(x) = ax
+ b$ defined over the range $-1 < x < 1$.

\begin{enumerate}[(i)]
  \item Consider now the new basis functions $\phi_1(x) =
    \frac{\sqrt{3}x}{2} + \frac{1}{2}$ and
    $\phi_2(x) = \frac{\sqrt{3}x}{2} - \frac{1}{2}$.
    Represent the functions $\phi_1(x)$ and $\phi_2(x)$ in a
    two-dimensional diagram with orthogonal axes
    corresponding to the functions $\psi_1(x)$ and $\psi_2(x)$
    respectively. \label{first}

    \boxedanswer{

      Express $f(x)$ in terms of $\psi_1$ and $\psi_2$:

      \begin{align*}
        1 &= \overbrace{\sqrt2}^b\psi_1(x) \\
        x &= \overbrace{\sqrt\frac{2}{3}}^a\psi_2(x)
      \end{align*}

      In the $[\psi_1, \psi_2]$ space, $f(x) = b\cdot 1 + a\cdot x$
      can be represented as $\left[b\sqrt{2}, a\sqrt{\frac{2}{3}}\right]$.

      For $\phi_1$ and $\phi_2$:

      \begin{align*}
        \phi_1(x) &= \overbrace{\frac{\sqrt{3}}{2}}^{a}x +
        \overbrace{\frac{1}{2}}^{b} \\
        \phi_2(x) &=
        \overbrace{\frac{\sqrt{3}}{2}}^{a}x
        \overbrace{-\frac{1}{2}}^{b}
      \end{align*}

      $\phi_1:$ $\left[\frac{\sqrt{2}}{2}, \frac{\sqrt{2}}{2}\right]$

      $\phi_2:$ $\left[-\frac{\sqrt{2}}{2}, \frac{\sqrt{2}}{2}\right]$

      \centering

      \input{src/fig_4_10_3.tex}

    }

    \pagebreak

  \item Construct the matrix that will transform a function in the
    “old” representation as a vector $
    \begin{bmatrix} c_1 \\ c_2
    \end{bmatrix}$
    into a new representation in terms of these new basis functions
    as a vector $
    \begin{bmatrix}d_1 \\ d_2
    \end{bmatrix}$, where an arbitrary function
    $f(x) = ax + b$
    is represented as the linear combination
    $f(x) = d_1 \phi_1(x) + d_2\phi_2(x)$. \label{second}
  \item Show that the matrix from part \ref{second} is unitary.
  \item Use the matrix of part \ref{second} to calculate the vector
    $
    \begin{bmatrix}d_1 \\ d_2
    \end{bmatrix}$
    for the specific example function $2x + 3$.
  \item Indicate the resulting vector on the same diagram as used for
    parts \ref{first}.
\end{enumerate}

\pagebreak

\section*{4.11.1}

For each of the following matrices, say whether or not it is unitary
and whether or not it is Hermitian.

\begin{enumerate}[(i)]
  \item $
    \begin{bmatrix} 1 & 0 \\ 0 & 1
    \end{bmatrix}$
  \item $
    \begin{bmatrix} 1 & i \\ -i & 1
    \end{bmatrix}$
  \item $
    \begin{bmatrix} i & 0 \\ 0 & i
    \end{bmatrix}$
  \item $
    \begin{bmatrix} 0 & 1 \\ i & 0
    \end{bmatrix}$
\end{enumerate}

\pagebreak

\section*{4.11.3 (i)}

Consider the Hermiticity of the following operators.

\begin{enumerate}[(i)]
  \item Prove that the momentum operator is Hermitian. For simplicity
    you may perform this proof
    for a one-dimensional system (i.e., only consider functions of
    $x$, and consider only the $\hat p_x$ operator).

    [Hints: Consider $\int_{-\infty}^\infty \psi_i^*(x)p_x\psi_j(x)dx$
      where the $\psi_n(x)$ are a complete orthonormal set.
      You may want to consider an integration by parts. Note that the
      $\psi_n(x)$ must vanish at $\pm \infty$,
    since otherwise they could not be normalized.]
\end{enumerate}

\pagebreak

\section*{5.1.1}

The Pauli spin matrices are quantum mechanical operators that operate
in a two-dimensional
Hilbert space, and can be written as

\begin{equation*}
  \hat\sigma_x =
  \begin{bmatrix} 0 & 1 \\ 1 & 0
  \end{bmatrix}, \hat\sigma_y =
  \begin{bmatrix} 0 & -i \\ i & 0
  \end{bmatrix}, \hat\sigma_z =
  \begin{bmatrix}
    1 & 0 \\ 0 & -1
  \end{bmatrix}
\end{equation*}

Find the commutation relations between each pair of these operators,
proving your answer by explicit
matrix multiplication, and simplifying the answers as much as possible.

\pagebreak

\section*{5.4.3}

Formally transform the momentum operator $\hat p_z$ into the momentum
basis using algebra similar to
that above for the transformation of the position operator into the
momentum basis.

\pagebreak

\end{document}

% Additional SI unit for Fahrenheit
\DeclareSIUnit\fahrenheit{\degree F}

\begin{document}

% Include title page
\input{./src/titlepage.tex}

\pagebreak

\section*{4.1.2}

Suppose that there are two quantum-mechanically measurable
quantities, $c$ with associated operator
$\hat C$ , and $d$ with associated operator $\hat D$ . In particular,
operator $\hat C$ has two eigenvectors $|\phi_1\rangle$ and $|\phi_2\rangle$,
and similarly operator $\hat D$ has two eigenvectors $|\psi_1\rangle$
and $|\psi_2\rangle$. The relation between the
eigenvectors is

\begin{align*}
  |\phi_1\rangle &= \frac{1}{5}(3|\psi_1\rangle + 4|\psi_2\rangle) \\
  |\phi_2\rangle &= \frac{1}{5}(4|\psi_1\rangle - 3|\psi_2\rangle)
\end{align*}

Suppose a measurement is made of the quantity $c$, and the system is
measured to be in state $|\phi_1\rangle$.
Then a measurement is made of quantity $d$, and following that the
quantity $c$ is again measured. What
is the probability (expressed as a fraction) that the system will be
found in state $|\psi_1\rangle$ on this second
measurement of $c$?

[Note: this is really a problem in quantum
  mechanical measurement discussed in
the previous Chapter, but is a good exercise in the use of the Dirac notation.]

\boxedanswer{

  \begin{align*}
    \hat C |\phi_1\rangle &= c|\phi_1\rangle \\
    \hat C |\phi_2\rangle &= c|\phi_2\rangle \\
    \hat D |\psi_1\rangle &= d|\phi_1\rangle \\
    \hat D |\psi_2\rangle &= d|\phi_2\rangle \\
    | \phi_1 \rangle &= \frac{1}{5}(3|\psi_1\rangle + 4 | \psi_2 \rangle) \\
    |\psi_1\rangle &= \frac{5}{3}| \phi_1 \rangle - \frac{4}{3}|
    \psi_2 \rangle \\
    |\phi_2\rangle &= \frac{1}{5}(4|\psi_1\rangle - 3|\psi_2\rangle) \\
    |\psi_2\rangle &= \frac{4}{3}|\psi_1\rangle - \frac{5}{3}|\phi_2\rangle \\
    |\psi_1\rangle &= \frac{5}{3}| \phi_1 \rangle - \frac{4}{3}
    \left(\frac{4}{3}|\psi_1\rangle -
    \frac{5}{3}|\phi_2\rangle\right) \\
    |\psi_1\rangle &= \frac{5}{3}| \phi_1 \rangle
    - \frac{16}{9}|\psi_1\rangle
    + \frac{20}{9}|\phi_2\rangle \\
    \frac{25}{9}|\psi_1\rangle
    &= \frac{5}{3}| \phi_1 \rangle
    + \frac{20}{9}|\phi_2\rangle \\
    |\psi_1\rangle
    &= \frac{3}{5}| \phi_1 \rangle
    + \frac{4}{5}|\phi_2\rangle \\
    |\psi_2\rangle &= \frac{4}{3}\left( \frac{3}{5}| \phi_1 \rangle
    + \frac{4}{5}|\phi_2\rangle\right) - \frac{5}{3}|\phi_2\rangle \\
    |\psi_2\rangle &= \frac{4}{5}| \phi_1 \rangle
    - \frac{3}{5}|\phi_2\rangle \\
  \end{align*}

  The probability of getting state $\psi_1$ after measuring $d$,
  given that the system is in state $\phi_1$.:

  \begin{align*}
    P(\psi_1) &= |\langle\psi_1|\phi_1\rangle|^2 =
    \left(\frac{3}{5}\overbrace{\langle\psi_1|\psi_1\rangle}^1 + \frac{4}{5}
    \cancel{\langle\psi_1| \psi_2 \rangle}\right)^2 \\
    P(\psi_1) &= \frac{9}{25}
  \end{align*}

}

\boxedanswer{
  The probability of getting state $\psi_2$ after measuring $d$,
  given that the system is in state $\phi_1$.:

  \begin{align*}
    P(\psi_2) &= |\langle\psi_2|\phi_1\rangle|^2 =
    \left(\frac{3}{5}\cancel{\langle\psi_2|\psi_1\rangle} + \frac{4}{5}
    \overbrace{\langle\psi_2| \psi_2 \rangle}^1\right)^2 \\
    P(\psi_2) &= \frac{16}{25}
  \end{align*}

  The probability of getting state $\phi_1$ given state $\psi_1$:

  \begin{align*}
    P(\phi_1|\psi_1) &= |\langle \phi_1|\psi_1 \rangle|^2 =
    \left(\frac{3}{5}\langle \phi_1| \phi_1 \rangle
    + \frac{4}{5}\langle \phi_1|\phi_2\rangle\right)^2 \\
    P(\phi_1|\psi_1) &= \left(\frac{3}{5}\overbrace{\langle \phi_1|
      \phi_1 \rangle}^1
    + \frac{4}{5}\cancel{\langle \phi_1|\phi_2\rangle}\right)^2 \\
    P(\phi_1|\psi_1) &= \frac{9}{25} \\
  \end{align*}

  The probability of getting state $\phi_1$ given state $\psi_2$:

  \begin{align*}
    P(\phi_1|\psi_2) &= |\langle \phi_1|\psi_2 \rangle|^2 =
    \left(\frac{4}{5}\overbrace{|\langle \phi_1| \phi_1 \rangle}^1
    - \frac{3}{5}\cancel{|\langle \phi_1|\phi_2\rangle}\right)^2 \\
    P(\phi_1|\psi_2) &= \frac{16}{25}
  \end{align*}

  Total probability:

  \begin{align*}
    P(\phi_2) &= P(\phi_1|\psi_1)P(\psi_1) + P(\phi_1|\psi_2)P(\psi_2) \\
    P(\phi_2) &= \left(\frac{9}{25}\right)^2 + \left(\frac{16}{25}\right)^2 \\
    \Aboxed{P(\phi_2) &= \frac{337}{625}}
  \end{align*}
}

\pagebreak

\section*{4.10.3}

Consider the orthonormal basis functions $\psi_1(x) = 1/\sqrt2$ and
$\psi_2(x) = \sqrt{\frac{3}{2}}x$
that are capable of representing any function of the form $f(x) = ax
+ b$ defined over the range $-1 < x < 1$.

\begin{enumerate}[(i)]
  \item Consider now the new basis functions $\phi_1(x) =
    \frac{\sqrt{3}x}{2} + \frac{1}{2}$ and
    $\phi_2(x) = \frac{\sqrt{3}x}{2} - \frac{1}{2}$.
    Represent the functions $\phi_1(x)$ and $\phi_2(x)$ in a
    two-dimensional diagram with orthogonal axes
    corresponding to the functions $\psi_1(x)$ and $\psi_2(x)$
    respectively. \label{first}

    \boxedanswer{

      Express $f(x)$ in terms of $\psi_1$ and $\psi_2$:

      \begin{align*}
        1 &= \overbrace{\sqrt2}^b\psi_1(x) \\
        x &= \overbrace{\sqrt\frac{2}{3}}^a\psi_2(x)
      \end{align*}

      In the $[\psi_1, \psi_2]$ space, $f(x) = b\cdot 1 + a\cdot x$
      can be represented as $\left[b\sqrt{2}, a\sqrt{\frac{2}{3}}\right]$.

      For $\phi_1$ and $\phi_2$:

      \begin{align*}
        \phi_1(x) &= \overbrace{\frac{\sqrt{3}}{2}}^{a}x +
        \overbrace{\frac{1}{2}}^{b} \\
        \phi_2(x) &=
        \overbrace{\frac{\sqrt{3}}{2}}^{a}x
        \overbrace{-\frac{1}{2}}^{b}
      \end{align*}

      $\phi_1:$ $\left[\frac{\sqrt{2}}{2}, \frac{\sqrt{2}}{2}\right]$

      $\phi_2:$ $\left[-\frac{\sqrt{2}}{2}, \frac{\sqrt{2}}{2}\right]$

      \centering

      \input{src/fig_4_10_3.tex}

    }

    \pagebreak

  \item Construct the matrix that will transform a function in the
    “old” representation as a vector $
    \begin{bmatrix} c_1 \\ c_2
    \end{bmatrix}$
    into a new representation in terms of these new basis functions
    as a vector $
    \begin{bmatrix}d_1 \\ d_2
    \end{bmatrix}$, where an arbitrary function
    $f(x) = ax + b$
    is represented as the linear combination
    $f(x) = d_1 \phi_1(x) + d_2\phi_2(x)$. \label{second}
  \item Show that the matrix from part \ref{second} is unitary.
  \item Use the matrix of part \ref{second} to calculate the vector
    $
    \begin{bmatrix}d_1 \\ d_2
    \end{bmatrix}$
    for the specific example function $2x + 3$.
  \item Indicate the resulting vector on the same diagram as used for
    parts \ref{first}.
\end{enumerate}

\pagebreak

\section*{4.11.1}

For each of the following matrices, say whether or not it is unitary
and whether or not it is Hermitian.

\begin{enumerate}[(i)]
  \item $
    \begin{bmatrix} 1 & 0 \\ 0 & 1
    \end{bmatrix}$
  \item $
    \begin{bmatrix} 1 & i \\ -i & 1
    \end{bmatrix}$
  \item $
    \begin{bmatrix} i & 0 \\ 0 & i
    \end{bmatrix}$
  \item $
    \begin{bmatrix} 0 & 1 \\ i & 0
    \end{bmatrix}$
\end{enumerate}

\pagebreak

\section*{4.11.3 (i)}

Consider the Hermiticity of the following operators.

\begin{enumerate}[(i)]
  \item Prove that the momentum operator is Hermitian. For simplicity
    you may perform this proof
    for a one-dimensional system (i.e., only consider functions of
    $x$, and consider only the $\hat p_x$ operator).

    [Hints: Consider $\int_{-\infty}^\infty \psi_i^*(x)p_x\psi_j(x)dx$
      where the $\psi_n(x)$ are a complete orthonormal set.
      You may want to consider an integration by parts. Note that the
      $\psi_n(x)$ must vanish at $\pm \infty$,
    since otherwise they could not be normalized.]
\end{enumerate}

\pagebreak

\section*{5.1.1}

The Pauli spin matrices are quantum mechanical operators that operate
in a two-dimensional
Hilbert space, and can be written as

\begin{equation*}
  \hat\sigma_x =
  \begin{bmatrix} 0 & 1 \\ 1 & 0
  \end{bmatrix}, \hat\sigma_y =
  \begin{bmatrix} 0 & -i \\ i & 0
  \end{bmatrix}, \hat\sigma_z =
  \begin{bmatrix}
    1 & 0 \\ 0 & -1
  \end{bmatrix}
\end{equation*}

Find the commutation relations between each pair of these operators,
proving your answer by explicit
matrix multiplication, and simplifying the answers as much as possible.

\pagebreak

\section*{5.4.3}

Formally transform the momentum operator $\hat p_z$ into the momentum
basis using algebra similar to
that above for the transformation of the position operator into the
momentum basis.

\pagebreak

\end{document}

% Additional SI unit for Fahrenheit
\DeclareSIUnit\fahrenheit{\degree F}

\begin{document}

% Include title page
\input{./src/titlepage.tex}

\pagebreak

\section*{4.1.2}

Suppose that there are two quantum-mechanically measurable
quantities, $c$ with associated operator
$\hat C$ , and $d$ with associated operator $\hat D$ . In particular,
operator $\hat C$ has two eigenvectors $|\phi_1\rangle$ and $|\phi_2\rangle$,
and similarly operator $\hat D$ has two eigenvectors $|\psi_1\rangle$
and $|\psi_2\rangle$. The relation between the
eigenvectors is

\begin{align*}
  |\phi_1\rangle &= \frac{1}{5}(3|\psi_1\rangle + 4|\psi_2\rangle) \\
  |\phi_2\rangle &= \frac{1}{5}(4|\psi_1\rangle - 3|\psi_2\rangle)
\end{align*}

Suppose a measurement is made of the quantity $c$, and the system is
measured to be in state $|\phi_1\rangle$.
Then a measurement is made of quantity $d$, and following that the
quantity $c$ is again measured. What
is the probability (expressed as a fraction) that the system will be
found in state $|\psi_1\rangle$ on this second
measurement of $c$?

[Note: this is really a problem in quantum
  mechanical measurement discussed in
the previous Chapter, but is a good exercise in the use of the Dirac notation.]

\boxedanswer{

  \begin{align*}
    \hat C |\phi_1\rangle &= c|\phi_1\rangle \\
    \hat C |\phi_2\rangle &= c|\phi_2\rangle \\
    \hat D |\psi_1\rangle &= d|\phi_1\rangle \\
    \hat D |\psi_2\rangle &= d|\phi_2\rangle \\
    | \phi_1 \rangle &= \frac{1}{5}(3|\psi_1\rangle + 4 | \psi_2 \rangle) \\
    |\psi_1\rangle &= \frac{5}{3}| \phi_1 \rangle - \frac{4}{3}|
    \psi_2 \rangle \\
    |\phi_2\rangle &= \frac{1}{5}(4|\psi_1\rangle - 3|\psi_2\rangle) \\
    |\psi_2\rangle &= \frac{4}{3}|\psi_1\rangle - \frac{5}{3}|\phi_2\rangle \\
    |\psi_1\rangle &= \frac{5}{3}| \phi_1 \rangle - \frac{4}{3}
    \left(\frac{4}{3}|\psi_1\rangle -
    \frac{5}{3}|\phi_2\rangle\right) \\
    |\psi_1\rangle &= \frac{5}{3}| \phi_1 \rangle
    - \frac{16}{9}|\psi_1\rangle
    + \frac{20}{9}|\phi_2\rangle \\
    \frac{25}{9}|\psi_1\rangle
    &= \frac{5}{3}| \phi_1 \rangle
    + \frac{20}{9}|\phi_2\rangle \\
    |\psi_1\rangle
    &= \frac{3}{5}| \phi_1 \rangle
    + \frac{4}{5}|\phi_2\rangle \\
    |\psi_2\rangle &= \frac{4}{3}\left( \frac{3}{5}| \phi_1 \rangle
    + \frac{4}{5}|\phi_2\rangle\right) - \frac{5}{3}|\phi_2\rangle \\
    |\psi_2\rangle &= \frac{4}{5}| \phi_1 \rangle
    - \frac{3}{5}|\phi_2\rangle \\
  \end{align*}

  The probability of getting state $\psi_1$ after measuring $d$,
  given that the system is in state $\phi_1$.:

  \begin{align*}
    P(\psi_1) &= |\langle\psi_1|\phi_1\rangle|^2 =
    \left(\frac{3}{5}\overbrace{\langle\psi_1|\psi_1\rangle}^1 + \frac{4}{5}
    \cancel{\langle\psi_1| \psi_2 \rangle}\right)^2 \\
    P(\psi_1) &= \frac{9}{25}
  \end{align*}

}

\boxedanswer{
  The probability of getting state $\psi_2$ after measuring $d$,
  given that the system is in state $\phi_1$.:

  \begin{align*}
    P(\psi_2) &= |\langle\psi_2|\phi_1\rangle|^2 =
    \left(\frac{3}{5}\cancel{\langle\psi_2|\psi_1\rangle} + \frac{4}{5}
    \overbrace{\langle\psi_2| \psi_2 \rangle}^1\right)^2 \\
    P(\psi_2) &= \frac{16}{25}
  \end{align*}

  The probability of getting state $\phi_1$ given state $\psi_1$:

  \begin{align*}
    P(\phi_1|\psi_1) &= |\langle \phi_1|\psi_1 \rangle|^2 =
    \left(\frac{3}{5}\langle \phi_1| \phi_1 \rangle
    + \frac{4}{5}\langle \phi_1|\phi_2\rangle\right)^2 \\
    P(\phi_1|\psi_1) &= \left(\frac{3}{5}\overbrace{\langle \phi_1|
      \phi_1 \rangle}^1
    + \frac{4}{5}\cancel{\langle \phi_1|\phi_2\rangle}\right)^2 \\
    P(\phi_1|\psi_1) &= \frac{9}{25} \\
  \end{align*}

  The probability of getting state $\phi_1$ given state $\psi_2$:

  \begin{align*}
    P(\phi_1|\psi_2) &= |\langle \phi_1|\psi_2 \rangle|^2 =
    \left(\frac{4}{5}\overbrace{|\langle \phi_1| \phi_1 \rangle}^1
    - \frac{3}{5}\cancel{|\langle \phi_1|\phi_2\rangle}\right)^2 \\
    P(\phi_1|\psi_2) &= \frac{16}{25}
  \end{align*}

  Total probability:

  \begin{align*}
    P(\phi_2) &= P(\phi_1|\psi_1)P(\psi_1) + P(\phi_1|\psi_2)P(\psi_2) \\
    P(\phi_2) &= \left(\frac{9}{25}\right)^2 + \left(\frac{16}{25}\right)^2 \\
    \Aboxed{P(\phi_2) &= \frac{337}{625}}
  \end{align*}
}

\pagebreak

\section*{4.10.3}

Consider the orthonormal basis functions $\psi_1(x) = 1/\sqrt2$ and
$\psi_2(x) = \sqrt{\frac{3}{2}}x$
that are capable of representing any function of the form $f(x) = ax
+ b$ defined over the range $-1 < x < 1$.

\begin{enumerate}[(i)]
  \item Consider now the new basis functions $\phi_1(x) =
    \frac{\sqrt{3}x}{2} + \frac{1}{2}$ and
    $\phi_2(x) = \frac{\sqrt{3}x}{2} - \frac{1}{2}$.
    Represent the functions $\phi_1(x)$ and $\phi_2(x)$ in a
    two-dimensional diagram with orthogonal axes
    corresponding to the functions $\psi_1(x)$ and $\psi_2(x)$
    respectively. \label{first}

    \boxedanswer{

      Express $f(x)$ in terms of $\psi_1$ and $\psi_2$:

      \begin{align*}
        1 &= \overbrace{\sqrt2}^b\psi_1(x) \\
        x &= \overbrace{\sqrt\frac{2}{3}}^a\psi_2(x)
      \end{align*}

      In the $[\psi_1, \psi_2]$ space, $f(x) = b\cdot 1 + a\cdot x$
      can be represented as $\left[b\sqrt{2}, a\sqrt{\frac{2}{3}}\right]$.

      For $\phi_1$ and $\phi_2$:

      \begin{align*}
        \phi_1(x) &= \overbrace{\frac{\sqrt{3}}{2}}^{a}x +
        \overbrace{\frac{1}{2}}^{b} \\
        \phi_2(x) &=
        \overbrace{\frac{\sqrt{3}}{2}}^{a}x
        \overbrace{-\frac{1}{2}}^{b}
      \end{align*}

      $\phi_1:$ $\left[\frac{\sqrt{2}}{2}, \frac{\sqrt{2}}{2}\right]$

      $\phi_2:$ $\left[-\frac{\sqrt{2}}{2}, \frac{\sqrt{2}}{2}\right]$

      \centering

      \input{src/fig_4_10_3.tex}

    }

    \pagebreak

  \item Construct the matrix that will transform a function in the
    “old” representation as a vector $
    \begin{bmatrix} c_1 \\ c_2
    \end{bmatrix}$
    into a new representation in terms of these new basis functions
    as a vector $
    \begin{bmatrix}d_1 \\ d_2
    \end{bmatrix}$, where an arbitrary function
    $f(x) = ax + b$
    is represented as the linear combination
    $f(x) = d_1 \phi_1(x) + d_2\phi_2(x)$. \label{second}
  \item Show that the matrix from part \ref{second} is unitary.
  \item Use the matrix of part \ref{second} to calculate the vector
    $
    \begin{bmatrix}d_1 \\ d_2
    \end{bmatrix}$
    for the specific example function $2x + 3$.
  \item Indicate the resulting vector on the same diagram as used for
    parts \ref{first}.
\end{enumerate}

\pagebreak

\section*{4.11.1}

For each of the following matrices, say whether or not it is unitary
and whether or not it is Hermitian.

\begin{enumerate}[(i)]
  \item $
    \begin{bmatrix} 1 & 0 \\ 0 & 1
    \end{bmatrix}$
  \item $
    \begin{bmatrix} 1 & i \\ -i & 1
    \end{bmatrix}$
  \item $
    \begin{bmatrix} i & 0 \\ 0 & i
    \end{bmatrix}$
  \item $
    \begin{bmatrix} 0 & 1 \\ i & 0
    \end{bmatrix}$
\end{enumerate}

\pagebreak

\section*{4.11.3 (i)}

Consider the Hermiticity of the following operators.

\begin{enumerate}[(i)]
  \item Prove that the momentum operator is Hermitian. For simplicity
    you may perform this proof
    for a one-dimensional system (i.e., only consider functions of
    $x$, and consider only the $\hat p_x$ operator).

    [Hints: Consider $\int_{-\infty}^\infty \psi_i^*(x)p_x\psi_j(x)dx$
      where the $\psi_n(x)$ are a complete orthonormal set.
      You may want to consider an integration by parts. Note that the
      $\psi_n(x)$ must vanish at $\pm \infty$,
    since otherwise they could not be normalized.]
\end{enumerate}

\pagebreak

\section*{5.1.1}

The Pauli spin matrices are quantum mechanical operators that operate
in a two-dimensional
Hilbert space, and can be written as

\begin{equation*}
  \hat\sigma_x =
  \begin{bmatrix} 0 & 1 \\ 1 & 0
  \end{bmatrix}, \hat\sigma_y =
  \begin{bmatrix} 0 & -i \\ i & 0
  \end{bmatrix}, \hat\sigma_z =
  \begin{bmatrix}
    1 & 0 \\ 0 & -1
  \end{bmatrix}
\end{equation*}

Find the commutation relations between each pair of these operators,
proving your answer by explicit
matrix multiplication, and simplifying the answers as much as possible.

\pagebreak

\section*{5.4.3}

Formally transform the momentum operator $\hat p_z$ into the momentum
basis using algebra similar to
that above for the transformation of the position operator into the
momentum basis.

\pagebreak

\end{document}

% Additional SI unit for Fahrenheit
\DeclareSIUnit\fahrenheit{\degree F}

\begin{document}

% Include title page
\input{./src/titlepage.tex}

\pagebreak

\section*{Problem 6.2.1}% {{{

Solve the problem above of an electron in a potential well with 3
units of field using the first two energy eigenfunctions of the well
without field as the finite basis subset. Give the energies (in the
dimensionless units) and explicit formulae for the normalized
eigenfunctions for the first two levels calculated by this method. Do
the algebra of this problem by hand, i.e., do not use mathematical
software to evaluate matrix elements or to solve for the eigenvalues
and eigenfunctions.

[Note: $\int_0^1(\xi-1/2)\sin(\pi\xi)\sin(2\pi\xi)d\xi=-(8/9\pi^2)$]

\begin{figure}[h]
  \centering
  \includegraphics[width=0.35\textwidth]{./assets/fig_6_2_1.png}
\end{figure}

\boxedanswer{

  The original Hamiltonian is:

  \begin{align*}
    V(z) &= e\mathrm{E}(z - L_z/2) \\
    \hat H &= -\frac{\hbar^2}{2m}\frac{d^2}{dz^2} + e\mathrm{E}(z - L_z/2) \\
  \end{align*}

  Some definitions:

  \begin{align*}
    E_1^\infty &=
    \left(\frac{\hbar^2}{2}\right)\left(\frac{\pi}{L_z}\right)^2 \\ \\
    \eta_n &= \frac{E_n}{E_1^\infty} \text{\qquad(Dimensionless
    eigenenergy of the $n$\textsuperscript{th} eigenstate.)}\\
    \mathrm{E}_o &= \frac{E_o}{eL_z} \\
    \mathrm{f} &= \mathrm{\frac{E}{E_o}} \text{\qquad Dimensionless field} \\
    \xi &= \frac{z}{L_z} \text{\qquad Dimensionless length}
  \end{align*}

  $E_n$ is the well eigenenergy, $E_1^\infty$ is the confinement energy of
  the first state of the infinite well.

  The Hamiltonian becomes:

  \begin{equation*}
    \hat H = -\frac{1}{\pi^2}\frac{d^2}{d\xi^2} + f(\xi - 1/2)
  \end{equation*}

  The unperturbed Schr\"odinger equation:

  \begin{equation*}
    \hat H_o \psi_n = \varepsilon_n\psi_n
  \end{equation*}

}

\pagebreak

\boxedanswer{

  And the general solution:

  \begin{equation*}
    \psi_n(\xi) = \sqrt{2}\sin(n\pi\xi)
  \end{equation*}

  We can construct the Hamiltonian as a matrix:

  \begin{align*}
    H_{ij} &= -\frac{1}{\pi^2} \int_0^1 \psi_i^*(\xi)
    \frac{d^2}{d\xi^2} \psi_j(\xi) d\xi + f\int_0^1 \psi_i^*(\xi)(\xi
    - 1/2)\phi_j(\xi) d\xi
  \end{align*}

  By taking the basis as the first three eignfunctions, we range $i$ and $j$
  from 1 to 2.

  \begin{align*}
    H_{1,1} &= -\frac{1}{\pi^2} \int_0^1 \psi_i^*(\xi) \frac{d^2}{d\xi^2}
    \psi_j(\xi) d\xi + f\int_0^1 \psi_i^*(\xi)(\xi - 1/2)\psi_j(\xi) d\xi \\
    &
    \begin{aligned}
      \psi_1(\xi) &= \psi_1^*(\xi) = \sqrt{2}\sin(\pi\xi) \\
      \frac{d^2}{d\xi^2} \psi_j(\xi) &= -\sqrt{2}\pi^2\sin(\pi\xi) \\
    \end{aligned} \\
    H_{1,1} &= -\frac{1}{\pi^2} \int_0^1 \left(\sqrt{2}\sin(\pi\xi)\right)
    \left(-\sqrt{2}\pi^2\sin(\pi\xi)\right) d\xi +
    f\int_0^1 (\xi - 1/2)\left(\sqrt{2}\sin(\pi\xi)\right)^2 d\xi \\
    H_{1,1} &= 2\pi^2\frac{1}{\pi^2} \int_0^1 \sin^2(\pi\xi)
    d\xi + 2f\int_0^1 (\xi - 1/2)\sin^2(\pi\xi)
    d\xi \\
    &
    \begin{aligned}
      2f\int_0^1 \overbrace{(\xi - 1/2)}^{\text{odd}}\overbrace{\sin^2(\pi\xi)^2
      }^{\text{even}}d\xi &= 0 \\
      2\int_0^1 \sin^2(\pi\xi) d\xi &=
      2\int_0^1\frac{1}{2}\left[1 - \cos(2\pi\xi) \right]d\xi \\
      2\int_0^1 \sin^2(\pi\xi) d\xi &=
      \left[\xi - \frac{1}{2\pi}\sin(2\pi\xi)\right]_0^1 \\
      2\int_0^1 \sin^2(\pi\xi) d\xi &= 1 \\
    \end{aligned} \\ H_{2,2} &= -\frac{1}{\pi^2} \int_0^1
    \psi_2^*(\xi) \frac{d^2}{d\xi^2}
    \psi_2(\xi) d\xi + f\int_0^1 \psi_2^*(\xi)(\xi - 1/2)\psi_2(\xi) d\xi \\
    &
    \begin{aligned}
      \frac{d^2}{d\xi^2} \psi_2(\xi) &= -4\sqrt{2}\pi^2\sin(2\pi\xi) \\
    \end{aligned} \\
    H_{2,2} &= 8\int_0^1\sin^2(2\pi\xi)
    d\xi+\cancel{2f\int_0^1(\xi-1/2)\sin^2(2\pi\xi)d\xi}\\
    H_{2,2} &= 4\int_0^1\left[1-\cos(4\pi\xi)\right] \\
    H_{2,2} &= 4\left[\xi-\frac{1}{4\pi}\sin(4\pi\xi)\right]_0^1 \\
    H_{2,2} &= 4 \\
    H_{1,2} &= -\frac{1}{\pi^2} \int_0^1 \psi_i^*(\xi) \frac{d^2}{d\xi^2}
    \psi_j(\xi) d\xi + f\int_0^1 \psi_i^*(\xi)(\xi - 1/2)\psi_j(\xi) d\xi \\
    &
    \begin{aligned}
      \psi_i^*(\xi) &= \sqrt{2}\sin(\pi\xi) \\
      \psi_j(\xi) &= \sqrt{2}\sin(2\pi\xi) \\
      \frac{d^2}{d\xi^2} \psi_j(\xi) &=
      -4\sqrt{2}\pi^2\sin(2\pi\xi) \\
    \end{aligned}
  \end{align*}

}

\boxedanswer{
  \begin{align*}
    H_{1,2} &= -\frac{1}{\pi^2} \int_0^1 \left(\sqrt{2}\sin(\pi\xi)\right)
    \left(-4\sqrt{2}\pi^2\sin(2\pi\xi)\right)d\xi+
    f\int_0^1 \left(\sqrt{2}\sin(\pi\xi)\right)(\xi - 1/2)
    \left(\sqrt{2}\sin(2\pi\xi)\right) d\xi \\
    H_{1,2} &= 8\underbrace{\int_0^1 \overbrace{\sin(\pi\xi)}^{\text{even}}
    \overbrace{\sin(2\pi\xi)}^{\text{odd}}d\xi}_0 -
    2\overbrace{f}^3\overbrace{\int_0^1 (\xi - 1/2)\sin(\pi\xi)
    \sin(2\pi\xi)}^{\frac{8}{9\pi^2}}d\xi \\
    H_{1,2} &= -\frac{16}{3\pi^2} \\
    H_{2,1} &= -\frac{1}{\pi^2} \int_0^1 \psi_i^*(\xi) \frac{d^2}{d\xi^2}
    \psi_j(\xi) d\xi + f\int_0^1 \psi_i^*(\xi)(\xi - 1/2)\psi_j(\xi) d\xi \\
    &
    \begin{aligned}
      \psi_i^*(\xi) &= \sqrt{2}\sin(2\pi\xi) \\
      \psi_j(\xi) &= \sqrt{2}\sin(\pi\xi) \\
      \frac{d^2}{d\xi^2} \psi_j(\xi) &=
      -\sqrt{2}\pi^2\sin(2\pi\xi) \\
    \end{aligned} \\
    H_{2,1} &= -\frac{1}{\pi^2} \int_0^1
    \left(\sqrt{2}\sin(2\pi\xi)\right)\left(-\sqrt{2}\pi^2\sin(\pi\xi)\right)d\xi+
    f\int_0^1 \left(\sqrt{2}\sin(2\pi\xi)\right)(\xi - 1/2)
    \left(\sqrt{2}\sin(\pi\xi)\right) d\xi \\
    H_{2,1} &= \underbrace{\int_0^1 \overbrace{\sin(2\pi\xi)}^{\text{odd}}
    \overbrace{\sin(\pi\xi)}^{\text{even}}d\xi}_0 -
    2\overbrace{f}^3\overbrace{\int_0^1 (\xi - 1/2)\sin(\pi\xi)
    \sin(2\pi\xi)}^{\frac{8}{9\pi^2}}d\xi \\
    H_{2,1} &= -\frac{16}{3\pi^2} \\
    &\hat H =
    \begin{bmatrix}
      1 & -\frac{16}{3\pi^2} \\
      -\frac{16}{3\pi^2} & 4 \\
    \end{bmatrix} \\
    \hat H \phi(\xi) &= \eta \phi(\xi) \\
    \phi(\eta) &=c_1\psi_1(\xi) + c_2\psi_2(\xi) \\
    |\hat H- \eta I|c &= 0 \\
    & a = \frac{16}{3\pi^2} \\
    0 &=
    \begin{bmatrix}
      1 - \eta & -a \\
      -a & 4 - \eta \\
    \end{bmatrix}
    \begin{bmatrix}
      c_1 \\ c_2
    \end{bmatrix} \\
    0 &= (1 - \eta)c_1 - ac_2 \\
    0 &= -ac_1 + (4 - \eta)c_2 \\
    c_2 &= \frac{1 - \eta}{a}c_1 \\
    c &\propto
    \begin{bmatrix}
      1 \\ \frac{1 - \eta}{a} \\
    \end{bmatrix} \\
    c &= \frac{1}{\sqrt{1 + \left(\frac{1 - \eta}{a}\right)^2}}
    \begin{bmatrix}
      1 \\ \frac{1 - \eta}{a} \\
    \end{bmatrix} \\
    \left|\hat H - \eta I\right| &= (1 - \eta)(4 - \eta) - a^2 = 0 \\
    0 &= \underbrace{1}_d\eta^2 \underbrace{- 5}_b\eta +
    \underbrace{4 - a^2}_c \\
    \eta &= \frac{-b \pm \sqrt{b^2 - 4dc}}{2d}
  \end{align*}
}

\boxedanswer{
  \begin{align*}
    \eta &= \frac{-(5) \pm \sqrt{(-5)^2 - 4(1)(4-a^2)}}{2d} \\
    \eta &= \frac{5}{2} \pm \sqrt{\frac{9}{4} + a^2} \\
    \Aboxed{\eta &= \frac{5}{2} \pm \sqrt{\frac{9}{4} +
    \left(\frac{16}{3\pi^2}\right)^2}} \\
    c_1 &= \frac{1}{\sqrt{1 + \left(\frac{1 - \frac{5}{2} - \sqrt{\frac{9}{4} +
    \left(\frac{16}{3\pi^2}\right)^2}}{\frac{16}{3\pi^2}}\right)^2}}
    \begin{bmatrix}
      1 \\ \frac{1 - \frac{5}{2} - \sqrt{\frac{9}{4} +
      \left(\frac{16}{3\pi^2}\right)^2}}{\frac{16}{3\pi^2}} \\
    \end{bmatrix} \\
    c_1 &=
    \begin{bmatrix}\frac{16\sqrt{2}}{\sqrt{81\pi ^4+9\pi
      ^2\sqrt{81\pi ^4+1024}+1024}}\\ \frac{-9\pi ^2-\sqrt{81\pi
      ^4+1024}}{\sqrt{2}\sqrt{81\pi ^4+9\pi ^2\sqrt{81\pi ^4+1024}+1024}}\\ 0
    \end{bmatrix} \\
    c_2 &= \frac{1}{\sqrt{1 + \left(\frac{1 - \frac{5}{2} + \sqrt{\frac{9}{4} +
    \left(\frac{16}{3\pi^2}\right)^2}}{\frac{16}{3\pi^2}}\right)^2}}
    \begin{bmatrix}
      1 \\ \frac{1 - \frac{5}{2} + \sqrt{\frac{9}{4} +
      \left(\frac{16}{3\pi^2}\right)^2}}{\frac{16}{3\pi^2}} \\
    \end{bmatrix} \\
    c_2 &=
    \begin{bmatrix}\frac{16\sqrt{2}}{\sqrt{81\pi ^4-9\pi
      ^2\sqrt{81\pi ^4+1024}+1024}}\\ \frac{-9\pi ^2+\sqrt{81\pi
      ^4+1024}}{\sqrt{2}\sqrt{81\pi ^4-9\pi ^2\sqrt{81\pi ^4+1024}+1024}}\\
    \end{bmatrix} \\
    &\boxed{\phi_1 =
      \begin{bmatrix}\frac{16\sqrt{2}}{\sqrt{81\pi ^4+9\pi
        ^2\sqrt{81\pi ^4+1024}+1024}}\\ \frac{-9\pi ^2-\sqrt{81\pi
        ^4+1024}}{\sqrt{2}\sqrt{81\pi ^4+9\pi ^2\sqrt{81\pi ^4+1024}+1024}}\\
      \end{bmatrix}
      \begin{bmatrix}
        \sqrt{2}\sin(\pi\xi) & \sqrt{2}\sin(2\pi\xi) \\
    \end{bmatrix}} \\
    &\boxed{\phi_2 =
      \begin{bmatrix}\frac{16\sqrt{2}}{\sqrt{81\pi ^4-9\pi
        ^2\sqrt{81\pi ^4+1024}+1024}}\\ \frac{-9\pi ^2+\sqrt{81\pi
        ^4+1024}}{\sqrt{2}\sqrt{81\pi ^4-9\pi ^2\sqrt{81\pi ^4+1024}+1024}}\\
      \end{bmatrix}
      \begin{bmatrix}
        \sqrt{2}\sin(\pi\xi) & \sqrt{2}\sin(2\pi\xi) \\
    \end{bmatrix}}
  \end{align*}
}

\pagebreak% }}}

\section*{Problem 6.3.2}

Consider an electron in a one-dimensional potential well of width $L_z$,
with infinitely high barriers on either side, and in which the
potential energy
inside the potential well is parabolic, of the form

\begin{equation}
  V(z)=u(z-L_z/2)^2
\end{equation}

where $u$ is a real constant. This potential is presumed to be small
compared to the energy E1 of the first confined state of a simple rectangular
potential well of the same width $L_z$.
[Note for interest: This kind of situation can arise in semiconductor
  structures, where the parabolic curvature comes from the
electrostatic potential of uniform background doping of the material.]

Find an approximate expression, valid in the limit of small $u$, for
the transition energy between the first and second allowed states of this well
in terms of $u$, $L_z$, and fundamental constants.

\boxedanswer{
  We will use time-inependent perterbation theory to solve this problem:

  The unperturbed system is as follows:
  \begin{equation*}
    \hat H_0 | \psi_n \rangle = E_n | \psi_n\rangle \\
  \end{equation*}

  The perturbation Hamiltonian can be written:

  \begin{equation*}
    \hat H_p = u(z - L_z/2)^2 \\
  \end{equation*}

  The Schr\"odinger equation now becomes:

  \begin{equation*}
    (\hat H_o + \gamma \hat H_p) | \phi \rangle = E | \phi \rangle \\
  \end{equation*}

  We expand $|\phi\rangle$ in a power series of $\gamma$:

  \begin{align*}
    |\phi\rangle &= |\phi^{(0)}\rangle + \gamma |\phi^{(1)}\rangle +
    \gamma^2|\phi^{(2)}\rangle + \gamma^3|\phi^{(3)}\rangle+\cdots \\
    E &= E^{(0)} + \gamma E^{(1)} + \gamma^2E^{(2)} +\gamma^3E^{(3)} + \cdots
  \end{align*}

  Substituting:

  \begin{multline*}
    (\hat H_o + \gamma \hat H_p)\left(|\phi^{(0)}\rangle + \gamma
      |\phi^{(1)}\rangle +
    \gamma^2|\phi^{(2)}\rangle + \gamma^3|\phi^{(3)}\rangle+\cdots\right)  = \\
    \left(E^{(0)} + \gamma E^{(1)} + \gamma^2E^{(2)} +\gamma^3E^{(3)} + \cdots
    \right)\left(|\phi^{(0)}\rangle + \gamma |\phi^{(1)}\rangle +
    \gamma^2|\phi^{(2)}\rangle + \gamma^3|\phi^{(3)}\rangle+\cdots  \right) \\
  \end{multline*}

  Comparing terms that are in 0\textsuperscript{th} order in $\gamma$:

  \begin{align*}
    \hat H_0 | \phi^{(0)} \rangle &= E^{(0)} | \phi^{(0)} \rangle
  \end{align*}

  The is the unperturbed Hamiltonian, so $\phi^{(0)} = |\psi_m\rangle$.

  \begin{align*}
    \hat H_0 | \psi_m \rangle &= E_m | \psi_m \rangle
  \end{align*}

  Then the terms in 1\textsuperscript{st} and 2\textsuperscript{nd}
  order in $\gamma$:

  \begin{align*}
    \hat H_o | \phi^{(1)} \rangle + \hat H_p|\psi_m\rangle &=
    E_m | \phi^{(1)} \rangle + E^{(1)} \psi_m \rangle \\
    \hat H_o | \phi^{(2)}\rangle + \hat H_p|\phi^{(1)}\rangle &=
    E_m| \phi^{(2)} \rangle + E^{(1)} | \phi^{(1)}\rangle +
    E^{(2)}|\psi_m\rangle \\
  \end{align*}

}

\boxedanswer{

  Rearranging:

  \begin{align*}
    (\hat H_o - E_m) |\psi_m \rangle &= 0 \\
    (\hat H_o - E_m) | \phi^{(1)} \rangle &=
    (E^{(1)} - \hat H_p) |\phi^{(1)} \rangle \\
    (\hat H_o - E_m) |\phi^{(2)}\rangle &= (E^{(1)} - \hat H_p ) |
    \phi^{(1)} \rangle + E^{(2)} |\psi_m\rangle
  \end{align*}

  We then solve for the first-order correction eigenenergy:

  \begin{align*}
    \langle \psi_m |(\hat H_o - E_m) | \phi^{(1)} \rangle &=
    \langle \psi_m |(E^{(1)} - \hat H_p) |\phi^{(1)} \rangle \\
    \langle \psi_m | \hat H_o - E_m | \phi^{(1)} \rangle &=
    (\langle \psi_m | \hat H_o - E_m ) | \phi^{(1)} \rangle =
    \langle \psi_m | (E_m - E_m ) | \phi^{(1)} \rangle = 0 \\
    &= \langle \psi_m | E^{(1)} - \hat H_p | \psi_m\rangle
    = E^{(1)} - \langle \psi_m | \hat H_p | \psi_m\rangle \\
    E^{(1)} &= \langle \psi_m | \hat H_p | \psi_m\rangle
  \end{align*}

  Solve for the first-order correction by substituting into the first-order
  equation and multiply by a particular $\psi_i$:

  \begin{align*}
    | \phi^{(1)} \rangle &= \sum_n a_n^{(1)} | \psi_n \rangle \\
    \langle \psi_i | \hat H_o - E_m | \phi^{(1)} \rangle &= (E_i - E_m)
    \langle \psi_i | \phi^{(1)} \rangle = (E_i - E_m) a_i^{(1)} \\
    &= \langle \psi_i | E^{(1)} - \hat H_p | \psi_m \rangle = E^{(1)}
    \langle \psi_i | \psi_m \rangle - \langle \psi_i | \hat H_p |
    \psi_m \rangle \\
  \end{align*}

  Assuming non-degeneracy and $i \ne m$:

  \begin{align*}
    a_i^{(1)} &=  \frac{\langle \psi_i | \hat H_p |
    \psi_m \rangle}{E_m - E_i}
  \end{align*}

  For $i = m$:

  \begin{align*}
    (E_m - E_m)a_m^{(1)} &= 0 a_m^{(1)} \\
    &= E^{(1)} - \langle \psi_m | \hat H_p | \psi_m \rangle
    = E^{(1)} - E^{(1)} = 0
  \end{align*}

  For convenience, we can choose $a_m^{(1)} = 0$. Therefore:

  \begin{equation*}
    \langle \psi_m | \phi^{(j)} \rangle = 0
  \end{equation*}

  Hence:

  \begin{equation*}
    | \phi^{(1)} \rangle = \sum_{n \ne m} \frac{\langle \psi_n | \hat
    H_p | \psi_m \rangle}{E_m - E_n} | \psi_n \rangle
  \end{equation*}

  For this problem , we chose the basis functions to be the first two
  eigenfunctions for the infinite well.

  \begin{align*}
    \psi_m &= \sqrt{2}\sin\left(\frac{m\pi z}{L_z}\right) \\
    \Delta E_{1\rightarrow 2} &= E_2 - E_1 \\
    &
    \begin{aligned}
      E^{(1)} &= \langle \psi_m | \hat H_p | \psi_m \rangle \\
      E^{(1)} &= \int_0^{L_z} \sqrt{2} \sin\left(\frac{m\pi z}{L_z}\right)
      u(z - L_z/2)^2\sqrt{2}\sin\left(\frac{m\pi z}{L_z}\right)dz
    \end{aligned}
  \end{align*}

}

\boxedanswer{
  \begin{align*}
    E^{(1)} &= 2\int_0^{L_z}\sin\left(\frac{m\pi z}{L_z}\right)
    u(z - L_z/2)^2\sin\left(\frac{m\pi z}{L_z}\right)dz \\
    E^{(1)} &= 2u\int_0^{L_z}\sin^2\left(\frac{m\pi z}{L_z}\right)
    (z - L_z/2)^2dz \\
    E^{(1)} &= 2u\int_0^{L_z}\sin^2\left(\frac{m\pi z}{L_z}\right)
    (z^2 - zL_z + (L_z/2)^2)dz \\
    E^{(1)}_1 &= 2u\int_0^{L_z}z^2\sin^2\left(\frac{\pi z}{L_z}\right) dz
    - 2uL_z\int_0^{L_z}z\sin^2\left(\frac{\pi z}{L_z}\right) dz
    + \frac{2uL_z^2}{4}\int_0^{L_z}\sin^2\left(\frac{\pi z}{L_z}\right) dz \\
    &
    \begin{aligned}
      2u\int_0^{L_z}z^2\sin^2\left(\frac{\pi z}{L_z}\right) dz &=
      \frac{\left(2\pi^{2} - 3\right) L_{z}^{3} u}{6\pi^{2}} \\
      2uL_z\int_0^{L_z}z\sin^2\left(\frac{\pi z}{L_z}\right) dz &=
      \frac{L_{z}^{3} u}{2} \\
      \frac{2uL_z^2}{4}\int_0^{L_z}\sin^2\left(\frac{\pi z}{L_z}\right) dz &=
      \frac{L_{z}^{3} u}{4}
    \end{aligned} \\
    E^{(1)}_1 &= \frac{\left(2\pi^{2} - 3\right) L_{z}^{3} u}{6\pi^{2}}
    - \frac{L_{z}^{3} u}{2}
    + \frac{L_{z}^{3} u}{4} \\
    E^{(1)}_1 &= \frac{\left(\pi^{2} - 6\right) L_{z}^{3} u}{12\pi^{2}} \\
    E^{(1)}_2 &= 2u\int_0^{L_z}\sin^2\left(\frac{2\pi z}{L_z}\right)
    (z - L_z/2)^2dz \\
    E^{(1)}_2 &= 2u\int_0^{L_z}z^2\sin^2\left(\frac{2\pi z}{L_z}\right) dz
    - 2uL_z\int_0^{L_z}z\sin^2\left(\frac{2\pi z}{L_z}\right) dz
    + \frac{2uL_z^2}{4}\int_0^{L_z}\sin^2\left(\frac{2\pi z}{L_z}\right) dz \\
    &
    \begin{aligned}
      2u\int_0^{L_z}z^2\sin^2\left(\frac{2\pi z}{L_z}\right) dz &=
      \frac{\left(8\pi^{2} - 3\right) L_{z}^{3} u}{24\pi^{2}}\\
      2uL_z\int_0^{L_z}z\sin^2\left(\frac{2\pi z}{L_z}\right) dz &=
      \frac{L_{z}^{3} u}{2} \\
      \frac{2uL_z^2}{4}\int_0^{L_z}\sin^2\left(\frac{2\pi z}{L_z}\right) dz &=
      \frac{L_{z}^{3} u}{4}\\
    \end{aligned} \\
    E^{(1)}_2 &=\frac{\left(2\pi^{2} - 3\right) L_{z}^{3} u}{24\pi^{2}} \\
    \Aboxed{\Delta E^{(1)}_{1 \rightarrow 2} &= \frac{3L_{z}^{3} u}{8\pi^{2}}}
  \end{align*}
}

\pagebreak

\section*{Problem 7.1.2}

Consider a one-dimensional semiconductor potential well of width Lz with
potential barriers on either side approximated as being infinitely high. An
electron in this potential well is presumed to behave with an effective mass
$m_{eff}$. Initially, there is an electron in the lowest state of this
potential well. We want to use this semiconductor structure as a detector for
a very short electric field pulse. If the electron is found in the second
energy state after the pulse, the electron is presumed to be collected as
photocurrent by some mechanism, and the pulse is therefore detected.

To model this device, we presume that the electric field pulse F (t ) can be
approximated as a “half-cycle” pulse of length Δt , i.e., a pulse of the form

\begin{equation*}
  F(t) = F_0\sin\left(\frac{\pi t}{\Delta t} \right)
\end{equation*}

for times $t$ from $0$ to $\Delta t$, and zero for all other times.

\begin{enumerate}[(i)]
  \item Find an approximate expression, valid for sufficiently small field
    amplitude $F_o$, for the probability of finding the electron in its
    second state after the pulse.
  \item For a pulse of length $\Delta t = \SI{100}{\femto\second}$, and a
    \ch{GaAs} semiconductor structure with $m_{eff} = 0.07m_o$ and
    width $L_z = \SI{10}{\nano\meter}$, for what minimum electric field
    magnitude $F_o$ does this detector have at least a \SI{1}{\percent}
    chance of detecting the pulse?
  \item For a full cycle pulse (i.e., one of the form $F(t) =
      F_o\sin(2\pi t/\Delta t)$
      for times $t$ from $0$ to $\Delta t$, and zero for all other
    times), what is the probability of detecting the pulse with this detector?
    Justify your answer.
\end{enumerate}

\pagebreak

\end{document}
